
\documentclass[a4paper,14pt]{report} % можно использовать кегель 8-12, 14, 17 и 20 пунктов
\usepackage[russian]{babel} % задаёт русский как основной язык текста
\usepackage[T2A]{fontenc} % задаёт кириллическую кодировку шрифта
\usepackage{cmap} % обеспечивает нормальное копирование и поиск русского текста в pdf 
\usepackage[utf8]{inputenc} % определяет юникодную кодировку самого .tex-файла
\setcounter{secnumdepth}{0}
\usepackage{geometry} % задаёт поля 
\geometry{left=3cm} % левое — 3 см
\geometry{right= 1.5cm} % правое — 1,5 см
\geometry{top=2cm} % верхнее — 2 см
\geometry{bottom=2cm} % нижнее — 2 см
\usepackage{setspace} \onehalfspacing % задаёт «полуторный» межстрочный интервал 
\usepackage{indentfirst} % автоматически добавляет отступ в каждый новый абзац
\setlength\parindent{1.25cm}
\begin{document} 
\Large
\begin{titlepage}
    \begin{center}
        {\bf  МИНОБРНАУКИ РОССИИ\\
        САНКТ-ПЕТЕРБУРГСКИЙ ГОСУДАРСТВЕННЫЙ\\
        ЭЛЕКТРОТЕХНИЧЕСКИЙ УНИВЕРСТИТЕТ\\
        <<ЛЭТИ>> ИМ. В. И. УЛЬЯНОВА (ЛЕНИНА)\\
    
        }
    \end{center}
    \vfill
        {
        \begin{center}
            КУРСОВАЯ РАБОТА\\
            по дисциплине <<Теоретические основы электротехники>>\\
            Тема: <<исследование прохождения сигналов через линейную активную электрическую цепь>>\\
        \end{center}
        }
        \
    \vfill
        {\noindent\parbox{4cm}{Студент гр. 3114}  \hfill \parbox{3cm}{\rule{3cm}{0.15mm}} \hfill \parbox{4cm}{\raggedleft Злобин М. А.}\\}
        \parbox{4cm}{Преподаватель} \hfill \parbox{3cm}{\rule{3cm}{0.15mm}} \hfill \parbox{4cm}{\raggedleft Завьялов А. Е.} \\ 
        \center Санкт-Петербург\\ 2024
\end{titlepage}
\begin{center}
    {\bf АННОТАЦИЯ\\} 
\end{center}
    {
        \indent Линейные электрические цепи играют ключевую роль в усилении и обработке сигналов, 
    проходящих через них. Для анализа таких цепей применяются методы преобразования Лапласа, 
    разложения в ряды Фурье и спектрального анализа. Изучение линейных цепей и сигналов, 
    которые через них проходят, позволяет предсказывать поведение схем при воздействии на них периодических сигналов.
    }
\begin{center}
    {\bf SUMMARY}
\end{center}
Linear electrical circuits are essential for amplifying and\\ processing the signals passing through them. 
Methods such as Laplace transform, Fourier series decomposition, and spectrum analysis are used to analyze these circuits. 
Studying linear circuits and the signals that pass through them allows for predicting the behavior of the circuit when subjected to certain periodic signals.
\newpage
\tableofcontents
\newpage
\section{\centeringВВЕДЕНИЕ}
Цель курсовой работы – практическое освоение методов анализа искажений электрических сигналов, проходящих через линейные активные   RC~– цепи, а также рассмотрение вопросов проектирования активных RC – цепей по заданным передаточным функциям. 
В курсовой работе требуется выполнить следующие пункты: \\  
    \indent 1) найти по заданной передаточной функции реакцию активной RC-цепи при воздействии одиночного импульса; \\
    \indent 2) рассчитать переходную и импульсную характеристики активной цепи;  \\
    \indent 3) найти спектральные характеристики аналогового входного сигнала и частотные характеристики цепи;  \\
    \indent 4) вычислить 	установившуюся 	реакцию 	цепи 	при 	воздействии 
периодической последовательности импульсов;  \\
    \indent 5) рассчитать параметры элементов активной цепи по заданной передаточной функции.
\end{document}