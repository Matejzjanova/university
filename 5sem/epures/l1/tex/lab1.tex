\documentclass[a4paper,14pt ]{article} % можно использовать кегель 8-12, 14, 17 и 20 пунктов
\DeclareMathSizes{14}{14}{14}{14}
\usepackage{extsizes}
\usepackage{graphicx}
\graphicspath{{../}}
\usepackage[russian]{babel} % задаёт русский как основной язык текста
\usepackage[T2A]{fontenc} % задаёт кириллическую кодировку шрифта
\usepackage{cmap} % обеспечивает нормальное копирование и поиск русского текста в pdf 
\usepackage[utf8]{inputenc} % определяет юникодную кодировку самого .tex-файла
\setcounter{secnumdepth}{0}
\usepackage{mathptmx} 
\usepackage{fontspec}
\usepackage{subcaption}

\setmainfont{Times New Roman}


\usepackage{geometry} % задаёт поля 
\geometry{left=3cm} % левое — 3 см
\geometry{right= 1.5cm} % правое — 1,5 см
\geometry{top=2cm} % верхнее — 2 см
\geometry{bottom=2cm} % нижнее — 2 см
\usepackage{setspace} \onehalfspacing % задаёт «полуторный» межстрочный интервал 
\usepackage{indentfirst} % автоматически добавляет отступ в каждый новый абзац
\usepackage{amsmath,amsfonts,amssymb,amsthm,mathtools,mathtext, physics}
\usepackage{float}
\usepackage{array}
\usepackage{tabularx}
\usepackage{titlesec}
\usepackage{zref}
\titleformat{\section}{\centering\normalfont\bfseries}{\thesection.}{0.5em}{}
\titleformat{\subsection}{\centering\normalfont\bfseries}{\thesubsection.}{0.5em}{}  % Исправлено
\titleformat{\subsubsection}{\centering\normalfont\bfseries}{\thesubsubsection.}{0.5em}{}
\setlength\parindent{1.25cm}
\setcounter{secnumdepth}{3}
\begin{document}
\begin{titlepage}
    \thispagestyle{empty}
    \begin{center}
        {\bf  МИНОБРНАУКИ РОССИИ\\
        САНКТ-ПЕТЕРБУРГСКИЙ ГОСУДАРСТВЕННЫЙ\\
        ЭЛЕКТРОТЕХНИЧЕСКИЙ УНИВЕРСТИТЕТ\\
        <<ЛЭТИ>> ИМ. В. И. УЛЬЯНОВА (ЛЕНИНА)\\
        кафедра РС\\
    
        }
    \end{center}
    \vfill
        {
        \begin{center}
            \bfseries
            Отчет по лабораторной работе №1\\
            по дисциплине <<Электропреобразовательные устройства>>\\
            Тема: <<Построение схемы, заданной логическим выражением>>\\
        \end{center}
        }
        \
    \vfill
        {\noindent\parbox{4cm}{Студент гр. 3114}  \hfill \parbox{3cm}{\rule{3cm}{0.15mm}} \hfill \parbox{4cm}{\raggedleft Злобин М. А.}} \\\\
        \parbox{4cm}{Преподаватель} \hfill \parbox{3cm}{\rule{3cm}{0.15mm}} \hfill \parbox{5cm}{\raggedleft Матвеев А. В. \\ Самсонова Т. Е.} \\ 
        \center Санкт-Петербург
        
        2025
\end{titlepage}
\renewcommand{\thesubfigure}{\thefigure.\arabic{subfigure}} % 1.1, 1.2, 2.1, 2.2...
\captionsetup[subfigure]{labelformat=simple}
\setcounter{page}{2}
\section{Исследование однотактного выпрямителя}
Без фильтра:
\begin{figure}[H]
    \centering
    \begin{subfigure}{0.4\linewidth}
       \includegraphics[width=\textwidth]{i1.png} 
       \caption{Осциллограмма $i_1$}
    \end{subfigure}
    \begin{subfigure}{0.4\linewidth}
       \includegraphics[width=\textwidth]{i2.png} 
       \caption{Осциллограмма $i_2$}
    \end{subfigure}
    \begin{subfigure}{0.4\linewidth}
       \includegraphics[width=\textwidth]{uv.png} 
       \caption{Осциллограмма $u_\text{в}$}
    \end{subfigure}
    \\
    \begin{subfigure}{0.4\linewidth}
       \includegraphics[width=\textwidth]{ud1.png} 
       \caption{Осциллограмма $u_{D1}$} 
    \end{subfigure} 
\end{figure}
C емокстным фильтром:
\begin{figure}[H]
    \centering
    \begin{subfigure}{0.4\linewidth}
       \includegraphics[width=\textwidth]{ci1.png} 
       \caption{Осциллограмма $i_1$}
    \end{subfigure}
    \begin{subfigure}{0.4\linewidth}
       \includegraphics[width=\textwidth]{ci2.png} 
       \caption{Осциллограмма $i_2$}
    \end{subfigure}
    \begin{subfigure}{0.4\linewidth}
       \includegraphics[width=\textwidth]{cuv.png} 
       \caption{Осциллограмма $u_\text{в}$}
    \end{subfigure}
\end{figure}
С индуктивным фильтром:
\begin{figure}[H] 
    \centering
    \begin{subfigure}{0.4\linewidth}
       \includegraphics[width=\textwidth]{li1.png} 
       \caption{Осциллограмма $i_1$}
    \end{subfigure}
    \begin{subfigure}{0.4\linewidth}
       \includegraphics[width=\textwidth]{li2.png} 
       \caption{Осциллограмма $i_2$}
    \end{subfigure}
    \begin{subfigure}{0.4\linewidth}
       \includegraphics[width=\textwidth]{luv.png} 
       \caption{Осциллограмма $u_\text{в}$}
    \end{subfigure}
    \begin{subfigure}{0.4\linewidth}
       \includegraphics[width=\textwidth]{lud1.png} 
       \caption{Осциллограмма $u_{D1}$}
    \end{subfigure}
    \begin{subfigure}{0.4\linewidth}
       \includegraphics[width=\textwidth]{lun.png} 
       \caption{Осциллограмма $u_n$}
    \end{subfigure}
\end{figure}
Индуктивный фильтр с обратным диодом:

\begin{figure}[H] 
    \centering
    \begin{subfigure}{0.4\linewidth}
       \includegraphics[width=\textwidth]{ldi1.png} 
       \caption{Осциллограмма $i_1$}
    \end{subfigure}
    \begin{subfigure}{0.4\linewidth}
       \includegraphics[width=\textwidth]{ldi2.png} 
       \caption{Осциллограмма $i_2$}
    \end{subfigure}
    \begin{subfigure}{0.4\linewidth}
       \includegraphics[width=\textwidth]{lduv.png} 
       \caption{Осциллограмма $u_\text{в}$}
    \end{subfigure}
    \begin{subfigure}{0.4\linewidth}
       \includegraphics[width=\textwidth]{ldud.png} 
       \caption{Осциллограмма $u_{D1}$}
    \end{subfigure}
    \begin{subfigure}{0.4\linewidth}
       \includegraphics[width=\textwidth]{ldun.png} 
       \caption{Осциллограмма $u_n$}
    \end{subfigure}
\end{figure}
C Г-образным фильтром:
\begin{figure}[H] 
    \centering
    \begin{subfigure}{0.4\linewidth}
       \includegraphics[width=\textwidth]{gi1.png} 
       \caption{Осциллограмма $i_1$}
    \end{subfigure}
    \begin{subfigure}{0.4\linewidth}
       \includegraphics[width=\textwidth]{gi2.png} 
       \caption{Осциллограмма $i_2$}
    \end{subfigure}
    \begin{subfigure}{0.4\linewidth}
       \includegraphics[width=\textwidth]{guv.png} 
       \caption{Осциллограмма $u_\text{в}$}
    \end{subfigure}
    \begin{subfigure}{0.4\linewidth}
       \includegraphics[width=\textwidth]{gud1.png} 
       \caption{Осциллограмма $u_{D1}$}
    \end{subfigure}
    \begin{subfigure}{0.4\linewidth}
       \includegraphics[width=\textwidth]{gun.png} 
       \caption{Осциллограмма $u_n$}
    \end{subfigure}
\end{figure}
\begin{table}[H]
   \caption{Расчёт мощностей, коэффициента пульсации и габаритного коэффициента}
    \centering
\begin{tabular}{|c|c|c|c|c|c|c|}
    \hline
     \space & $P_1$ & $P_2$ & $P_0$ & $P_\text{ГАБ}$ & $k_\text{ГАБ}$ & $k_\text{п}$ \\ 
     \hline
     Без фильтра & 8 & 3.6 & 0.9 & 5.8 & 6.2 & 1.5 \\
     \hline
     C & 9.6 & 6 & 2.2 &7.8 & 3.5 & 0.7 \\
     \hline
     L & 6.6 & 0.9 & 0.07 & 3.8 & 50 & 1.21 \\
     \hline
     L с диодом & 6.7 & 1.5 &  0.47 & 4.12 & 8.7 & 0.35 \\
     \hline
     Г & 6.6 & 1.4 & 0.08 & 3.9 & 46 & 0.54 \\
    \hline
\end{tabular}


\end{table}

\begin{figure}[H]
    \centering
    \includegraphics[width=\linewidth]{UI1.jpg}
    \caption{График зависимости постоянной составляющей напряжения на нагрузке от постоянной составляющей тока нагрузки}
\end{figure}
\begin{table}[H]
    \centering
   \caption{Расчёт внутреннего сопротивления выпрямителя}
\begin{tabular}{|c|c|c|c|c|c|}

    \hline
     \space & БФ & С & L & L c диодом & Г\\ 
     \hline
     R, Ом & 67 & 68 & 94 & 40 & 69  \\
     \hline
\end{tabular}
\end{table}
\newpage
\section{Исследование двухтактного выпрямителя}
Без фильтра:
\begin{figure}[H]
    \centering
    \begin{subfigure}{0.4\linewidth}
       \includegraphics[width=\textwidth]{2i1.png} 
       \caption{Осциллограмма $i_1$}
    \end{subfigure}
    \begin{subfigure}{0.4\linewidth}
       \includegraphics[width=\textwidth]{2i2.png} 
       \caption{Осциллограмма $i_2$}
    \end{subfigure}
    \begin{subfigure}{0.4\linewidth}
       \includegraphics[width=\textwidth]{2uv.png} 
       \caption{Осциллограмма $u_\text{в}$}
    \end{subfigure}
    \\
    \begin{subfigure}{0.4\linewidth}
       \includegraphics[width=\textwidth]{2ud1.png} 
       \caption{Осциллограмма $u_{D1}$} 
    \end{subfigure} 
\end{figure}
C емокстным фильтром:
\begin{figure}[H]
    \centering
    \begin{subfigure}{0.4\linewidth}
       \includegraphics[width=\textwidth]{2ci1.png} 
       \caption{Осциллограмма $i_1$}
    \end{subfigure}
    \begin{subfigure}{0.4\linewidth}
       \includegraphics[width=\textwidth]{2ci2.png} 
       \caption{Осциллограмма $i_2$}
    \end{subfigure}
    \begin{subfigure}{0.4\linewidth}
       \includegraphics[width=\textwidth]{2cuv.png} 
       \caption{Осциллограмма $u_\text{в}$}
    \end{subfigure}
\end{figure}
С индуктивным фильтром:
\begin{figure}[H] 
    \centering
    \begin{subfigure}{0.4\linewidth}
       \includegraphics[width=\textwidth]{2li1.png} 
       \caption{Осциллограмма $i_1$}
    \end{subfigure}
    \begin{subfigure}{0.4\linewidth}
       \includegraphics[width=\textwidth]{2li2.png} 
       \caption{Осциллограмма $i_2$}
    \end{subfigure}
    \begin{subfigure}{0.4\linewidth}
       \includegraphics[width=\textwidth]{2luv.png} 
       \caption{Осциллограмма $u_\text{в}$}
    \end{subfigure}
    \begin{subfigure}{0.4\linewidth}
       \includegraphics[width=\textwidth]{2lud1.png} 
       \caption{Осциллограмма $u_{D1}$}
    \end{subfigure}
    \begin{subfigure}{0.4\linewidth}
       \includegraphics[width=\textwidth]{2lun.png} 
       \caption{Осциллограмма $u_n$}
    \end{subfigure}
\end{figure}
C Г-образным фильтром:
\begin{figure}[H] 
    \centering
    \begin{subfigure}{0.4\linewidth}
       \includegraphics[width=\textwidth]{2gi1.png} 
       \caption{Осциллограмма $i_1$}
    \end{subfigure}
    \begin{subfigure}{0.4\linewidth}
       \includegraphics[width=\textwidth]{2gi2.png} 
       \caption{Осциллограмма $i_2$}
    \end{subfigure}
    \begin{subfigure}{0.4\linewidth}
       \includegraphics[width=\textwidth]{2guv.png} 
       \caption{Осциллограмма $u_\text{в}$}
    \end{subfigure}
    \begin{subfigure}{0.4\linewidth}
       \includegraphics[width=\textwidth]{2gud1.png} 
       \caption{Осциллограмма $u_{D1}$}
    \end{subfigure}
    \begin{subfigure}{0.4\linewidth}
       \includegraphics[width=\textwidth]{2gun.png} 
       \caption{Осциллограмма $u_n$}
    \end{subfigure}
\end{figure}
\begin{table}[H]
   \caption{Расчёт мощностей, коэффициента пульсации и габаритного коэффициента}
    \centering
\begin{tabular}{|c|c|c|c|c|c|c|}
    \hline
     \space & $P_1$ & $P_2$ & $P_0$ & $P_\text{ГАБ}$ & $k_\text{ГАБ}$ & $k_\text{п}$ \\ 
     \hline
     Без фильтра & 8.8 &  0.32 & 3.6 & 4.5 & 1.28 & 0.75 \\
     \hline
     C & 9.9 & 0.4 & 4.98 & 5.11 & 1.02 & 0.29 \\
     \hline
     L & 7.5 & 0.324& 1.92 &  3.94 & 2.05 & 0.09\\
     \hline
     Г & 7.7 & 0.32 & 1.88 & 4.04 & 2.14 & 0.03 \\
    \hline
\end{tabular}
\end{table}
\begin{figure}[H]
    \centering
    \includegraphics[width=\linewidth]{UI2.jpg}
    \caption{График зависимости постоянной составляющей напряжения на нагрузке от постоянной составляющей тока нагрузки}
\end{figure}
\begin{table}[H]
   \caption{Расчёт внутреннего сопротивления выпрямителя}
    \centering
\begin{tabular}{|c|c|c|c|c|}
    \hline
     \space & БФ & С & L & Г\\ 
     \hline
     R, Ом & 8.3 & 21 & 45 & 69  \\
     \hline
\end{tabular}
\end{table}
\newpage
\section{Исследование сглаживающих фильтров}
\begin{figure}[H]
    \centering
    \includegraphics[width=\linewidth]{K.jpg}
    \caption{График коэффициента пульсаций напряжения на нагрузке от постоянной составляющей тока}
\end{figure}
\newpage
Определим коэффициент сглаживания пульсаций при максимальном токе нагрузки:
\begin{table}[H]
   \caption{Определение коэффициента сглаживания пульсаций.}
   \centering
\begin{tabular}{|c|c|c|}
    \hline
     \space & L & Г \\  
     \hline
     $k_\text{вх}$ & 1.19 & 1.2 \\
     \hline
     $k_\text{вых}$ & 0.3 & 0.14 \\
     \hline
     S & 3.97 & 8.57 \\
     \hline
\end{tabular}
\end{table}
\section{Вывод} 
В ходе работы были исследованы однофазные и двухфазные выпрямители с различными фильтрами: индуктивным, емкостным, Г-образным, в различных комбинациях со включением обратных диодов, 
построены зависимости постоянной составляющей тока в нагрузке от значения нагрузки и построена зависимость коэффициента пульсаций напряжения на нагрузке от постоянной
составляющей тока.
\end{document}