\documentclass[a4paper,14pt ]{article} % можно использовать кегель 8-12, 14, 17 и 20 пунктов
\DeclareMathSizes{14}{14}{14}{14}
\usepackage{subcaption}
\usepackage{caption}

\captionsetup[figure]{name=Рисунок}
\usepackage{multicol}
\captionsetup[subfigure]{labelformat=simple}
\usepackage{extsizes}
\usepackage{graphicx}
\graphicspath{{../}}
\usepackage[russian]{babel} % задаёт русский как основной язык текста
\usepackage[T2A]{fontenc} % задаёт кириллическую кодировку шрифта
\usepackage{cmap} % обеспечивает нормальное копирование и поиск русского текста в pdf 
\usepackage[utf8]{inputenc} % определяет юникодную кодировку самого .tex-файла
\setcounter{secnumdepth}{0}
\usepackage{mathptmx} 
\usepackage{fontspec}

\usepackage{caption}
\usepackage[labelsep=endash, name=Рисунок]{subcaption}
\usepackage[labelsep=endash]{caption}

\captionsetup[figure]{name=Рисунок}
\captionsetup[subfigure]{labelformat=simple}
\setmainfont{Times New Roman}



\usepackage{geometry} % задаёт поля 
\geometry{left=3cm} % левое — 3 см
\geometry{right= 1.5cm} % правое — 1,5 см
\geometry{top=2cm} % верхнее — 2 см
\geometry{bottom=2cm} % нижнее — 2 см
\usepackage{setspace} \onehalfspacing % задаёт «полуторный» межстрочный интервал 
\usepackage{indentfirst} % автоматически добавляет отступ в каждый новый абзац
\usepackage{amsmath,amsfonts,amssymb,amsthm,mathtools,mathtext, physics}
\usepackage{float}
\usepackage{array}
\usepackage{tabularx}
\usepackage{titlesec}
\usepackage{zref}
\titleformat{\section}{\centering\normalfont\bfseries}{\thesection.}{0.5em}{}
\titleformat{\subsection}{\centering\normalfont\bfseries}{\thesubsection.}{0.5em}{}  % Исправлено
\titleformat{\subsubsection}{\centering\normalfont\bfseries}{\thesubsubsection.}{0.5em}{}
\setlength\parindent{1.25cm}
\setcounter{secnumdepth}{3}
\begin{document}
\begin{titlepage}
    \thispagestyle{empty}
    \begin{center}
        {\bf  МИНОБРНАУКИ РОССИИ\\
        САНКТ-ПЕТЕРБУРГСКИЙ ГОСУДАРСТВЕННЫЙ\\
        ЭЛЕКТРОТЕХНИЧЕСКИЙ УНИВЕРСТИТЕТ\\
        <<ЛЭТИ>> ИМ. В. И. УЛЬЯНОВА (ЛЕНИНА)\\
        кафедра РС\\
    
        }
    \end{center}
    \vfill
        {
        \begin{center}
            \bfseries
            Отчет по лабораторной работе №4\\
            по дисциплине <<Электропреобразовательные устройства>>\\
            Тема: <<Импульсные стабилизаторы>>\\
        \end{center}
        }
        \
    \vfill
        {\noindent\parbox{4cm}{Студенты гр. 3114}  \hfill \parbox{3cm}{\rule{3cm}{0.15mm}} \hfill \parbox{4cm}{\raggedleft Злобин М. А. \\ Тимошко С. И. \\ Федулова Е. В.}} \\\\
        \parbox{4cm}{Преподаватель} \hfill \parbox{3cm}{\rule{3cm}{0.15mm}} \hfill \parbox{5cm}{\raggedleft Матвеев А. В. \\ Самсонова Т. Е.} \\ 
        \center Санкт-Петербург
        
        2025
\end{titlepage}
\renewcommand{\thesubfigure}{\thefigure.\arabic{subfigure}} % 1.1, 1.2, 2.1, 2.2...
\captionsetup[subfigure]{labelformat=simple}
\setcounter{page}{2}
\section{Исследование преоборазователя постоянного напряжения понижающего типа}
\subsection{Осциллограммы напряжений и токов на разных элементах схемы при частоте 10 кГц}
\begin{multicols}{2}
  \centering
\begin{figure}[H]
    \includegraphics[width=\linewidth]{./ug.png}
    \caption{Напряжение на генераторе}
  \end{figure}
  \begin{figure}[H]
    \includegraphics[width=\linewidth]{./uce10.png}
    \caption{Напряжение между коллектором и эмиттером регулировочного тразистора}
  \end{figure}
  \begin{figure}[H]
    \includegraphics[width=\linewidth]{./ul10.png}
    \caption{Напряжение на дросселе}
  \end{figure}
  \begin{figure}[H]
    \includegraphics[width=\linewidth]{./ic10.png}
    \caption{Коллекторный ток}
  \end{figure}
  \begin{figure}[H]
    \includegraphics[width=\linewidth]{./ud10.png}
    \caption{Напряжение на диоде}
  \end{figure}
  \begin{figure}[H]
    \includegraphics[width=\linewidth]{./il110.png}
    \caption{Дроссельный ток}
  \end{figure}
\end{multicols}

\begin{figure}[H]
    \centering
    \includegraphics[width=0.5\linewidth]{./un10.png}
    \caption{Напряжение на нагрузке}
\end{figure}

\subsection{Осциллограммы дроссельного и коллекторного токов}
\begin{multicols}{2}
  \begin{figure}[H]
    \includegraphics[width=\linewidth]{./1il.png}
    \caption{Дроссельный ток, частота 1 кГц}
  \end{figure}
  \begin{figure}[H]
    \includegraphics[width=\linewidth]{./2il.png}
    \caption{Дроссельный ток, частота 2 кГц}
  \end{figure}
  \begin{figure}[H]
    \includegraphics[width=\linewidth]{./1ic.png}
    \caption{Коллекторный ток, частота 1 кГц}
  \end{figure}
  \begin{figure}[H]
    \includegraphics[width=\linewidth]{./2ic.png}
    \caption{Коллекторный ток, частота 2 кГц}
  \end{figure}
  \begin{figure}[H]
    \includegraphics[width=\linewidth]{./5il.png}
    \caption{Дроссельный ток, частота 5 кГц}
  \end{figure}
  \begin{figure}[H]
    \includegraphics[width=\linewidth]{./10il.png}
    \caption{Дроссельный ток, частота 10 кГц}
  \end{figure}
  \begin{figure}[H]
    \includegraphics[width=\linewidth]{./5ic.png}
    \caption{Коллекторный ток, частота 5 кГц}
  \end{figure}
  \begin{figure}[H]
    \includegraphics[width=\linewidth]{./10ic.png}
    \caption{Коллекторный ток, частота 10 кГц}
  \end{figure}
\end{multicols}

На рисунках 4, 6, 10 и других видны выбросы коллекторного тока,
которые можно объяснить наличием паразитной емкости в транзисторе,
которая разряжается при его октрытии (рисунок 4 и остальные), что может привести к пробою, а скачки дроссельного тока 
происходят при резком напряжений на транзисторе и диоде.
\subsection{Построим регулировочные и нагрузочные характеристики для понижающего 
стабилизатора}
  \begin{figure}[H]
    \centering
    \includegraphics[width=0.7\linewidth]{./llIk.jpg}
    \caption{Нагрузочная характеристика коллекторного тока}
  \end{figure}
  \begin{figure}[H]
    \centering
    \includegraphics[width=0.7\linewidth]{./llIn.jpg}
    \caption{Нагрузочная характеристика тока нагрузки}
  \end{figure}
  \begin{figure}[H]
    \centering
    \includegraphics[width=0.7\linewidth]{./llg.jpg}
    \caption{Нагрузочная характеристика длительности импульсов коллекторного тока транзистора}
  \end{figure}
  \begin{figure}[H]
    \centering
    \includegraphics[width=0.7\linewidth]{./adjlik.jpg}
    \caption{Регулировочная характеристика коллекторного тока}
  \end{figure}
  \begin{figure}[H]
    \centering
    \includegraphics[width=0.7\linewidth]{./adjlun.jpg}
    \caption{Регулировочная характеристика напряжения нагрузки}
  \end{figure}
  \begin{figure}[H]
    \centering
    \includegraphics[width=0.7\linewidth]{./adjlg.jpg}
    \caption{Регулировочная характеристика длительности импульсов коллекторного тока транзистора}
  \end{figure}
\section{Исследование преобразователя постоянного напряжения инвертирующего типа}
\subsection{Осциллограммы напряжений и токов на различных элементах схемы при частоте 10 кГц}
\begin{multicols}{2}
  \centering
\begin{figure}[H]
    \includegraphics[width=\linewidth]{./ug.png}
    \caption{Напряжение на генераторе}
  \end{figure}
  \begin{figure}[H]
    \includegraphics[width=\linewidth]{./2uce.png}
    \caption{Напряжение между коллектором и эмиттером регулировочного тразистора}
  \end{figure}
  \begin{figure}[H]
    \includegraphics[width=\linewidth]{./2ul.png}
    \caption{Напряжение на дросселе}
  \end{figure}
  \begin{figure}[H]
    \includegraphics[width=\linewidth]{./22ic.png}
    \caption{Коллекторный ток}
  \end{figure}
  \begin{figure}[H]
    \includegraphics[width=\linewidth]{./2uvd.png}
    \caption{Напряжение на диоде}
  \end{figure}
  \begin{figure}[H]
    \includegraphics[width=\linewidth]{./22il.png}
    \caption{Дроссельный ток}
  \end{figure}
  \end{multicols}
\subsection{Построим регулировочные и нагрузочные характеристики для инвертирующего преобразователя
}
  \begin{figure}[H]
    \centering
    \includegraphics[width=0.7\linewidth]{./liIk.jpg}
    \caption{Нагрузочная характеристика коллекторного тока}
  \end{figure}
  \begin{figure}[H]
    \centering
    \includegraphics[width=0.7\linewidth]{./liIn.jpg}
    \caption{Нагрузочная характеристика тока нагрузки}
  \end{figure}
  \begin{figure}[H]
    \centering
    \includegraphics[width=0.7\linewidth]{./lig.jpg}
    \caption{Нагрузочная характеристика длительности импульсов коллекторного тока транзистора}
  \end{figure}
  \begin{figure}[H]
    \centering
    \includegraphics[width=0.7\linewidth]{./adjiik.jpg}
    \caption{Регулировочная характеристика коллекторного тока}
  \end{figure}
  \begin{figure}[H]
    \centering
    \includegraphics[width=0.7\linewidth]{./adjiun.jpg}
    \caption{Регулировочная характеристика напряжения нагрузки}
  \end{figure}
  \begin{figure}[H]
    \centering
    \includegraphics[width=0.7\linewidth]{./adjig.jpg}
    \caption{Регулировочная характеристика длительности импульсов коллекторного тока транзистора}
  \end{figure}
\section{Исследование импульсного стабилизатора постоянного напряжения
понижающего типа в режиме ШИМ}
Построим нагрузочные характеристики импульсного стабилизатора в режиме ШИМ
  \begin{figure}[H]
    \centering
    \includegraphics[width=0.7\linewidth]{./lpwmin.jpg}
    \caption{Нагрузочная характеристика тока нагрузки}
  \end{figure}
  \begin{figure}[H]
    \centering
    \includegraphics[width=0.7\linewidth]{./lpwmg.jpg}
    \caption{Нагрузочная характеристика длительности импульсов коллекторного тока транзистора}
  \end{figure}
Построим зависимости напряжения нагрузки, времени импульса коллекторого тока и его относительной длительности для импульсного стабилизатора в режиме ШИМ
  \begin{figure}[H]
    \centering
    \includegraphics[width=0.7\linewidth]{./inpwmun.jpg}
    \caption{Зависимость напряжения на нагрузке от входного напряжения}
  \end{figure}
  \begin{figure}[H]
    \centering
    \includegraphics[width=0.7\linewidth]{./inpwmt.jpg}
    \caption{Зависимость периода импульса коллекторого тока от входного напряжения}
  \end{figure}
  \begin{figure}[H]
    \centering
    \includegraphics[width=0.7\linewidth]{./inpwmg.jpg}
    \caption{Зависимость cкважности импульса коллекторого тока от входного напряжения}
  \end{figure}
\section{Исследование импульсного стабилизатора постоянного напряжения понижающего типа в релейном режиме}
Построим нагрузочные характеристики импульсного стабилизатора в режиме ШИМ
  \begin{figure}[H]
    \centering
    \includegraphics[width=0.5\linewidth]{./relin.jpg}
    \caption{Нагрузочная характеристика тока нагрузки}
  \end{figure}
  \begin{figure}[H]
    \centering
    \includegraphics[width=0.5\linewidth]{./relg.jpg}
    \caption{Нагрузочная характеристика длительности импульсов коллекторного тока транзистора}
  \end{figure}
Построим зависимости напряжения нагрузки, времени импульса коллекторого тока и его относительной длительности для импульсного стабилизатора в релейном режиме 
  \begin{figure}[H]
    \centering
    \includegraphics[width=0.5\linewidth]{./inrelun.jpg}
    \caption{Зависимость напряжения на нагрузке от входного напряжения}
  \end{figure}
  \begin{figure}[H]
    \centering
    \includegraphics[width=0.5\linewidth]{./inrelt.jpg}
    \caption{Зависимость периода импульса коллекторого тока от входного напряжения}
  \end{figure}
  \begin{figure}[H]
    \centering
    \includegraphics[width=0.7\linewidth]{./inrelg.jpg}
    \caption{Зависимость cкважности импульса коллекторого тока от входного напряжения}
  \end{figure}
\section{Вывод}
Входе работы были исследованы различные импульсные стабилизаторы и преоборазователи напряжений:
понижающий стабилизатор, инвертирующий преоборазователь, преоборазователь в режимах ШИМ и релейном, построены их нагрузочные и 
регулировочные характеристики
\end{document}
