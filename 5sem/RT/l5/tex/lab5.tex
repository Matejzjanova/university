\documentclass[a4paper,14pt ]{article} % можно использовать кегель 8-12, 14, 17 и 20 пунктов
\DeclareMathSizes{14}{14}{14}{14}
\usepackage{extsizes}
\usepackage{graphicx}
\graphicspath{{../}}
\usepackage[russian]{babel} % задаёт русский как основной язык текста
\usepackage[T2A]{fontenc} % задаёт кириллическую кодировку шрифта
\usepackage{cmap} % обеспечивает нормальное копирование и поиск русского текста в pdf 
\usepackage[utf8]{inputenc} % определяет юникодную кодировку самого .tex-файла
\setcounter{secnumdepth}{0}
\usepackage{mathptmx} 
\usepackage{fontspec}
\usepackage{subcaption}

\setmainfont{Times New Roman}


\usepackage{geometry} % задаёт поля 
\geometry{left=3cm} % левое — 3 см
\geometry{right= 1.5cm} % правое — 1,5 см
\geometry{top=2cm} % верхнее — 2 см
\geometry{bottom=2cm} % нижнее — 2 см
\usepackage{setspace} \onehalfspacing % задаёт «полуторный» межстрочный интервал 
\usepackage{indentfirst} % автоматически добавляет отступ в каждый новый абзац
\usepackage{amsmath,amsfonts,amssymb,amsthm,mathtools,mathtext, physics}
\usepackage{float}
\usepackage{array}
\usepackage{tabularx}
\usepackage{titlesec}
\usepackage{zref}
\titleformat{\section}{\centering\normalfont\bfseries}{\thesection.}{0.5em}{}
\titleformat{\subsection}{\centering\normalfont\bfseries}{\thesubsection.}{0.5em}{}  % Исправлено
\titleformat{\subsubsection}{\centering\normalfont\bfseries}{\thesubsubsection.}{0.5em}{}
\setlength\parindent{1.25cm}
\setcounter{secnumdepth}{3}
\begin{document}
\begin{titlepage}
    \thispagestyle{empty}
    \begin{center}
        {\bf  МИНОБРНАУКИ РОССИИ\\
        САНКТ-ПЕТЕРБУРГСКИЙ ГОСУДАРСТВЕННЫЙ\\
        ЭЛЕКТРОТЕХНИЧЕСКИЙ УНИВЕРСТИТЕТ\\
        <<ЛЭТИ>> ИМ. В. И. УЛЬЯНОВА (ЛЕНИНА)\\
        кафедра РС\\
    
        }
    \end{center}
    \vfill
        {
        \begin{center}
            \bfseries
            Отчет по лабораторной работе №5\\
            по дисциплине <<Радиотехнические цепи и сигналы>>\\
            Тема: <<Исследование прохождения амплитудно-модулированных сигналов через избирательные цепи>>\\
        \end{center}
        }
        \
    \vfill
        {\noindent\parbox{4cm}{Студент гр. 3114}  \hfill \parbox{3cm}{\rule{3cm}{0.15mm}} \hfill \parbox{4cm}{\raggedleft Злобин М. А.}} \\\\
        \parbox{4cm}{Преподаватель} \hfill \parbox{3cm}{\rule{3cm}{0.15mm}} \hfill \parbox{5cm}{\raggedleft Пышкин С. И. } \\ 
        \center Санкт-Петербург
        
        2025
\end{titlepage}
\renewcommand{\thesubfigure}{\thefigure.\arabic{subfigure}} % 1.1, 1.2, 2.1, 2.2...
\section{{Описание лабораторной установки}}

\begin{figure}[H]
    \centering
    \includegraphics[width=0.5\linewidth]{lm.png}
    \caption{Схема лабораторного макета}
    \label{fig:1}
\end{figure}

Лабораторный макет (рис. \ref{fig:1}) содержит 2 идентичных колебательных
контура и переключатель, с помощью которого 2 контура соединяются меж-
ду собой через переменный конденсатор. Сигнал от внешнего источника по-
ступает на вход первого контура. Выходной сигнал регистрируется либо на
выходе первого контура, либо на выходе второго контура в зависимости от
положения переключателя.
\section{Исследование АЧХ одиночного колебательного контура}

\begin{figure}[H]
    \centering
    \includegraphics[width=0.9\linewidth]{apc.jpg}
    \caption{АЧХ одиночного колебательного контура}
    \label{fig:4}
\end{figure}
\section{Исследование АЧХ системы связанных контуров}
\begin{figure}[H]
    \centering
    \includegraphics[width=0.9\linewidth]{apc2.jpg}
    \caption{АЧХ системы связанных контуров (минимальная связь)}
    \label{fig:3}
\end{figure}
\begin{figure}[H]
    \centering
    \includegraphics[width=0.9\linewidth]{apc3.jpg}
    \caption{АЧХ системы связанных контуров (максимальная связь)}
    \label{fig:2}
\end{figure}
\section{Исследование прохождения АМК с тональной модуляцией через
колебательный контур}
Найдем коэффициент модуляции m в спектралной области, посчитав отношение боковых гармоник к центральной (предварительно переведя
из dBm в Вольты).
\begin{figure}[H]
    \centering
    \begin{subfigure}{0.4\linewidth}
        \includegraphics[width=\linewidth]{example1.jpg}
        \caption{Значение боковой гармоники на частоте 4 кГц}
        \label{fig:5}
    \end{subfigure}
    \begin{subfigure}{0.4\linewidth}
        \includegraphics[width=\linewidth]{example2.png}
        \caption{Значение центрально гармоники на частоте 4 кГц}
        \label{fig:5}
    \end{subfigure}
\end{figure}
\begin{figure}[H]
    \centering
    \includegraphics[width=0.9\linewidth]{m1.jpg}
    \caption{Зависимость коэффициента модуляции от частоты модулирующего колебания}
    \label{fig:6}
\end{figure}
\section{Исследование прохождения АМК с тональной модуляцией через систему связанных контуров}
Аналогично посчитает зависимость m от частоты модулирующего колебания для системы связанных контуров:
\begin{figure}[H]
    \centering
    \includegraphics[width=0.9\linewidth]{m2.jpg}
    \caption{Зависимость коэффициента модуляции от частоты модулирующего колебания}
    \label{fig:7}
\end{figure}
\section{Исследование искажений АМК с тональной модуляцией при про-
хождении через систему связанных контуров}
Зафиксированные осцилограммы сигнала с перемодуляцией:
\begin{figure}[H]
    \centering
    \includegraphics[width=0.9\linewidth]{b2.png}
    \caption{Осциллограмма АМК с перемодуляцией при f = 410 кГЦ}
    \label{fig:8}
\end{figure}
\begin{figure}[H]
    \centering
    \includegraphics[width=0.9\linewidth]{bieniya.png}
    \caption{Осциллограмма АМК с перемодуляцией при f = 550 кГЦ}
    \label{fig:8}
\end{figure}
\section{Вывод}
В ходе лабораторной работы были исследованы характеристики одиночного колебательного контура и системы из двух связанных rолебательных контуров,
изучены преобразования амплитудно-модулированных колебаний при прохождении через частотно-избирательные цепи. Были построены графики коэффициента модуляции для случая 
одиночного контура (рис. \ref{fig:6}) и для случая системы связанных контуров с максимальной емкостью связи (рис. \ref{fig:7}) 
\end{document}