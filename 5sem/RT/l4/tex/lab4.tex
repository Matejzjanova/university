% \documentclass[a4paper,14pt ]{article} % можно использовать кегель 8-12, 14, 17 и 20 пунктов
% \DeclareMathSizes{14}{14}{14}{14}
% \usepackage{extsizes}
% \usepackage{graphicx}
% \graphicspath{{../}}
% \usepackage[russian]{babel} % задаёт русский как основной язык текста
% \usepackage[T2A]{fontenc} % задаёт кириллическую кодировку шрифта
% \usepackage{cmap} % обеспечивает нормальное копирование и поиск русского текста в pdf 
% \usepackage[utf8]{inputenc} % определяет юникодную кодировку самого .tex-файла
% \setcounter{secnumdepth}{0}
% \usepackage{mathptmx} 
% \usepackage{fontspec}
% \usepackage{subcaption}
%
% \setmainfont{Times New Roman}
%
%
% \usepackage{geometry} % задаёт поля 
% \geometry{left=3cm} % левое — 3 см
% \geometry{right= 1.5cm} % правое — 1,5 см
% \geometry{top=2cm} % верхнее — 2 см
% \geometry{bottom=2cm} % нижнее — 2 см
% \usepackage{setspace} \onehalfspacing % задаёт «полуторный» межстрочный интервал 
% \usepackage{indentfirst} % автоматически добавляет отступ в каждый новый абзац
% \usepackage{amsmath,amsfonts,amssymb,amsthm,mathtools,mathtext, physics}
% \usepackage{float}
% \usepackage{array}
% \usepackage{tabularx}
% \usepackage{titlesec}
% \usepackage{zref}
% \titleformat{\section}{\centering\normalfont\bfseries}{\thesection.}{0.5em}{}
% \titleformat{\subsection}{\centering\normalfont\bfseries}{\thesubsection.}{0.5em}{}  % Исправлено
% \titleformat{\subsubsection}{\centering\normalfont\bfseries}{\thesubsubsection.}{0.5em}{}
% \setlength\parindent{1.25cm}
% \setcounter{secnumdepth}{3}
\documentclass[a4paper,14pt]{report} % можно использовать кегель 8-12, 14, 17 и 20 пунктов
\usepackage[russian]{babel} % задаёт русский как основной язык текста
\usepackage[T2A]{fontenc} % задаёт кириллическую кодировку шрифта
\usepackage{cmap} % обеспечивает нормальное копирование и поиск русского текста в pdf 
\usepackage[utf8]{inputenc} % определяет юникодную кодировку самого .tex-файла

\usepackage{geometry} % задаёт поля 
\geometry{left=3cm} % левое — 3 см
\geometry{right= 1.5cm} % правое — 1,5 см
\geometry{top=2cm} % верхнее — 2 см
\geometry{bottom=2cm} % нижнее — 2 см
\usepackage{setspace} \onehalfspacing % задаёт «полуторный» межстрочный интервал 
\usepackage{indentfirst} % автоматически добавляет отступ в каждый новый абзац

\begin{document}
\begin{titlepage}
    \thispagestyle{empty}
    \begin{center}
        {\bf  МИНОБРНАУКИ РОССИИ\\
        САНКТ-ПЕТЕРБУРГСКИЙ ГОСУДАРСТВЕННЫЙ\\
        ЭЛЕКТРОТЕХНИЧЕСКИЙ УНИВЕРСТИТЕТ\\
        <<ЛЭТИ>> ИМ. В. И. УЛЬЯНОВА (ЛЕНИНА)\\
        кафедра РС\\
    
        }
    \end{center}
    \vfill
        {
        \begin{center}
            \bfseries
            Отчет по лабораторной работе №4\\
            по дисциплине <<Радиотехнические цепи и сигналы>>\\
            Тема: <<Исследование прохождения амплитудно-модулированных сигналов через избирательные цепи>>\\
        \end{center}
        }
        \
    \vfill
        {\noindent\parbox{4cm}{Студент гр. 3114}  \hfill \parbox{3cm}{\rule{3cm}{0.15mm}} \hfill \parbox{4cm}{\raggedleft Злобин М. А.}} \\\\
        \parbox{4cm}{Преподаватель} \hfill \parbox{3cm}{\rule{3cm}{0.15mm}} \hfill \parbox{5cm}{\raggedleft Пышкин С. И. } \\ 
        \center Санкт-Петербург
        
        2025
\end{titlepage}
  \section{Определение масштаба графиков}
  Определим масштаб по вертикали, исходя из рисунков \ref{fig:1}, \ref{fig:2}, \ref{fig:3}:
  \begin{figure}[H]
    \centering
    \includegraphics[width=0.8\linewidth]{t1pd.jpg}
    \caption{Плотность распределения начальной фазы треугольного сигнала}
    \label{fig:1}
  \end{figure}
  \begin{figure}[H]
    \centering
    \includegraphics[width=0.8\linewidth]{t1.jpg}
    \caption{Осциллограмма треугольного сигнала}
    \label{fig:2}
  \end{figure}
  \begin{figure}[H]
    \centering
    \includegraphics[width=0.8\linewidth]{t1df.jpg}
    \caption{Функция распределения треугольного сигнала}
    \label{fig:3}
  \end{figure}

Размах сигнала = $\Delta U = 0.32  \text{ B}$

Масштаб по горизонтали:
\begin{equation}
  (U_2 - U_1) L= 0.32 \text{ В}
\end{equation}

Изображение графика плотности вероятности по вертикали
занимает $\Delta L = 0.8$ клеток.
Масштаб по вертикали:
\begin{equation}
  \frac{1}{(U_2 - U_1)\Delta L} = 3.9 \text{\space} \frac{\text{1}}{\text{В}\cdot\text{дел}}
\end{equation}
\section{Исследование треугольного сигнала}
Треугольный сигнал при $U_m = 0.8$ B:
  \begin{figure}[H]
    \centering
    \includegraphics[width=0.8\linewidth]{t2pd.jpg}
    \caption{Плотность распределения начальной фазы треугольного сигнала}
    \label{fig:11}
  \end{figure}
  \begin{figure}[H]
    \centering
    \includegraphics[width=0.8\linewidth]{t2.jpg}
    \caption{Осциллограмма треугольного сигнала}
    \label{fig:12}
  \end{figure}
  \begin{figure}[H]
    \centering
    \includegraphics[width=0.8\linewidth]{t2df.jpg}
    \caption{Функция распределения треугольного сигнала}
    \label{fig:13}
  \end{figure}
Видно, что ростом амплитуды функция распределения и плотность вероятности пропорционально "растягиваются", что соответвует большему диапазону значений,
которые может принимать начальная фаза.
\section{Исследование гармонического сигнала со случайной начальной фазой}
\begin{figure}[H]
  \centering
  \begin{subfigure}{0.4\linewidth}
    \includegraphics[width=\textwidth]{sinpd.jpg}
    \caption{Плотность распределения напряжения синусоидального сигнала со случайной НФ}
    \label{fig:21}
  \end{subfigure}
  \begin{subfigure}{0.4\linewidth}
    \includegraphics[width=\textwidth]{sin2pd.jpg}
    \caption{Плотность распределения напряжения синусоидального сигнала со случайной НФ}
    \label{fig:22}
  \end{subfigure}
  \begin{subfigure}{0.4\linewidth}
    \includegraphics[width=\textwidth]{sin2df.jpg}
    \caption{Функция распределения синусоидального сигнала}
    \label{fig:23}
  \end{subfigure}
  \begin{subfigure}{0.4\linewidth}
    \includegraphics[width=\textwidth]{sin2df.jpg}
    \caption{Функция распределения синусоидального сигнала}
    \label{fig:24}
  \end{subfigure}
  \begin{subfigure}{0.4\linewidth}
    \includegraphics[width=\textwidth]{sin2.jpg}
    \caption{Функция распределения синусоидального сигнала}
    \label{fig:25}
  \end{subfigure}
  \begin{subfigure}{0.4\linewidth}
    \includegraphics[width=\textwidth]{sin2.jpg}
    \caption{Функция распределения синусоидального сигнала}
    \label{fig:26}
  \end{subfigure}
\end{figure}
\section{Исследование шума с распределенем Гаусса}
\begin{figure}[H]
  \centering
  \begin{subfigure}{0.4\linewidth}
    \includegraphics[width=\textwidth]{noise1pd.jpg}
    \caption{Плотность распределения AWGN}
    \label{fig:31}
  \end{subfigure}
  \begin{subfigure}{0.4\linewidth}
    \includegraphics[width=\textwidth]{noise2pd.jpg}
    \caption{Плотность распределения AWGN}
    \label{fig:32}
  \end{subfigure}
  \begin{subfigure}{0.4\linewidth}
    \includegraphics[width=\textwidth]{noise1df.jpg}
    \caption{Функция распределения AWGN}
    \label{fig:33}
  \end{subfigure}
  \begin{subfigure}{0.4\linewidth}
    \includegraphics[width=\textwidth]{noise2df.jpg}
    \caption{Функция распределения AWGN}
    \label{fig:34}
  \end{subfigure}
\end{figure}
\section{Исследование шума с распределенем Рэлея}
\begin{figure}[H]
  \centering
  \begin{subfigure}{0.4\linewidth}
    \includegraphics[width=\textwidth]{nr1pd.jpg}
    \caption{Плотность распределения AWGN}
    \label{fig:41}
  \end{subfigure}
  \begin{subfigure}{0.4\linewidth}
    \includegraphics[width=\textwidth]{nr2pd.jpg}
    \caption{Плотность распределения AWGN}
    \label{fig:42}
  \end{subfigure}
  \begin{subfigure}{0.4\linewidth}
    \includegraphics[width=\textwidth]{nr1df.jpg}
    \caption{Функция распределения AWGN}
    \label{fig:43}
  \end{subfigure}
  \begin{subfigure}{0.4\linewidth}
    \includegraphics[width=\textwidth]{nr2df.jpg}
    \caption{Функция распределения AWGN}
    \label{fig:44}
  \end{subfigure}
\end{figure}
\section{Исследование сходимости к гауссовскому случайному процессу суммы независимых СП}
\begin{figure}[H]
  \centering
  \begin{subfigure}{0.4\linewidth}
    \includegraphics[width=\textwidth]{clt1.jpg}
    \caption{1 треугольный сигнал}
   \label{fig:51}
  \end{subfigure}
  \begin{subfigure}{0.4\linewidth}
    \includegraphics[width=\textwidth]{cltt1.jpg}
    \caption{1 гармонический сигнал}
    \label{fig:52}
  \end{subfigure}
  \begin{subfigure}{0.4\linewidth}
    \includegraphics[width=\textwidth]{clt2.jpg}
    \caption{2 треугольных сигнала}
    \label{fig:53}
  \end{subfigure}
  \begin{subfigure}{0.4\linewidth}
    \includegraphics[width=\textwidth]{cltt2.jpg}
    \caption{2 гармонических сигнала}
    \label{eig:54}
  \end{subfigure}
  \begin{subfigure}{0.4\linewidth}
    \includegraphics[width=\textwidth]{clt3.jpg}
    \caption{3 треугольных сигнала}
    \label{fig:55}
  \end{subfigure}
  \begin{subfigure}{0.4\linewidth}
    \includegraphics[width=\textwidth]{cltt3.jpg}
    \caption{3 гармонических сигнала}
    \label{fig:56}
  \end{subfigure}
  \begin{subfigure}{0.4\linewidth}
    \includegraphics[width=\textwidth]{clt4.jpg}
    \caption{4 треугольных сигнала}
  \end{subfigure}
  \begin{subfigure}{0.4\linewidth}
    \includegraphics[width=\textwidth]{cltt4.jpg}
    \caption{4 гармонических сигнала}
  \end{subfigure}
\end{figure}
\section*{Вывод}
В данной работе были проанализированы функции распределения и плотности вероятностей различных сигналов с разными диапазонами. Так, случайный треугольный сигнал демонстрирует равномерное распределение вероятности, тогда как гауссовский шум характеризуется нормальным (гауссовским) распределением. Также были рассмотрены гармонические сигналы и рэлеевский шум. Интересно отметить, что сумма треугольных сигналов быстрее сходится к гауссовскому распределению, что проявляется уже при объединении трёх треугольных сигналов. Это связано с тем, что плотность вероятности треугольных сигналов напоминает равномерное распределение, и при их сложении результат приближается к нормальному распределению. Наиболее быстрая сходимость наблюдается у рэлеевского шума с максимальной амплитудой.
\end{document}


