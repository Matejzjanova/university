% \documentclass[a4paper,14pt ]{article} % можно использовать кегель 8-12, 14, 17 и 20 пунктов
% \DeclareMathSizes{14}{14}{14}{14}
% \usepackage{extsizes}
% \usepackage{graphicx}
% \graphicspath{{../}}
% \usepackage[russian]{babel} % задаёт русский как основной язык текста
% \usepackage[T2A]{fontenc} % задаёт кириллическую кодировку шрифта
% \usepackage{cmap} % обеспечивает нормальное копирование и поиск русского текста в pdf 
% \usepackage[utf8]{inputenc} % определяет юникодную кодировку самого .tex-файла
% \setcounter{secnumdepth}{0}
% \usepackage{mathptmx} 
% \usepackage{fontspec}
% \usepackage{subcaption}
%
% \setmainfont{Times New Roman}
%
%
% \usepackage{geometry} % задаёт поля 
% \geometry{left=3cm} % левое — 3 см
% \geometry{right= 1.5cm} % правое — 1,5 см
% \geometry{top=2cm} % верхнее — 2 см
% \geometry{bottom=2cm} % нижнее — 2 см
% \usepackage{setspace} \onehalfspacing % задаёт «полуторный» межстрочный интервал 
% \usepackage{indentfirst} % автоматически добавляет отступ в каждый новый абзац
% \usepackage{amsmath,amsfonts,amssymb,amsthm,mathtools,mathtext, physics}
% \usepackage{float}
% \usepackage{array}
% \usepackage{tabularx}
% \usepackage{titlesec}
% \usepackage{zref}
% \titleformat{\section}{\centering\normalfont\bfseries}{\thesection.}{0.5em}{}
% \titleformat{\subsection}{\centering\normalfont\bfseries}{\thesubsection.}{0.5em}{}  % Исправлено
% \titleformat{\subsubsection}{\centering\normalfont\bfseries}{\thesubsubsection.}{0.5em}{}
% \setlength\parindent{1.25cm}
% \setcounter{secnumdepth}{3}
\documentclass[a4paper,14pt]{report} % можно использовать кегель 8-12, 14, 17 и 20 пунктов
\usepackage[russian]{babel} % задаёт русский как основной язык текста
\usepackage[T2A]{fontenc} % задаёт кириллическую кодировку шрифта
\usepackage{cmap} % обеспечивает нормальное копирование и поиск русского текста в pdf 
\usepackage[utf8]{inputenc} % определяет юникодную кодировку самого .tex-файла

\usepackage{geometry} % задаёт поля 
\geometry{left=3cm} % левое — 3 см
\geometry{right= 1.5cm} % правое — 1,5 см
\geometry{top=2cm} % верхнее — 2 см
\geometry{bottom=2cm} % нижнее — 2 см
\usepackage{setspace} \onehalfspacing % задаёт «полуторный» межстрочный интервал 
\usepackage{indentfirst} % автоматически добавляет отступ в каждый новый абзац

\begin{document}
\begin{titlepage}
    \thispagestyle{empty}
    \begin{center}
        {\bf  МИНОБРНАУКИ РОССИИ\\
        САНКТ-ПЕТЕРБУРГСКИЙ ГОСУДАРСТВЕННЫЙ\\
        ЭЛЕКТРОТЕХНИЧЕСКИЙ УНИВЕРСТИТЕТ\\
        <<ЛЭТИ>> ИМ. В. И. УЛЬЯНОВА (ЛЕНИНА)\\
        кафедра БЖД\\
    
        }
    \end{center}
    \vfill
        {
        \begin{center}
            \bfseries
            Отчет по лабораторной работе №5\\
            по дисциплине <<Безопасность жизнедеятельности>>\\
            Тема: <<Оказание первое доврачебной помощи
            >>\\
        \end{center}
        }
        \
    \vfill
        {\noindent\parbox{4cm}{Студенты гр. 3114}  \hfill \parbox{3cm}{\rule{3cm}{0.15mm}} \hfill \parbox{4cm}{\raggedleft Злобин М. А.\\ Тимошко С. И.}} \\\\
        \parbox{4cm}{Преподаватель} \hfill \parbox{3cm}{\rule{3cm}{0.15mm}} \hfill \parbox{5cm}{\raggedleft Демидович О. В. } \\ 
        \center Санкт-Петербург
        
        2025
\end{titlepage}

Цели работы 
\begin{itemize}
  \item Изучение алгоритмов оценки состояния пострадавшего
    , методов и приемов оказания первой неотложной доврачебной помощи;
  \item Отработка приемов, получение навыков оказания первой доврачебной неотложной помощи.
\end{itemize}
\section{Общие сведения}
Первая доврачебная неотложная помощь -- комплекс мероприятий, направленных
на спасение жизни и сохранения здоровья человека до прибытия медицинских работников.

Основыные задачи ПДНП:
\begin{itemize}
  \item Проведение мероприятий по ликвидации угрозы для пострадавшего;
  \item Предупреждение возможных осложнений
  \item Обеспечение благоприятных условий для траспортировки пострадавшего
\end{itemize}
\section{Ход работы и обработка результатов}
\section{Вывод}
В ходе работы были исследованы частотые характеристики шума, построены спектрограммы
фонового и имитации промышленного шумов. Также были исследована эффективность различных мероприятий
по шумоглушению: наибольшей эффективностью обладает применения шумопоглощения (кожух с шумопоглощаящим материалом)
и шумоотражения (стальной щит). Только примение данных средств позволило спектрограмме шума попасть ниже кривой ПС-75.
\end{document}


