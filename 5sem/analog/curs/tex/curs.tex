
\documentclass[a4paper,14pt]{report} % можно использовать кегель 8-12, 14, 17 и 20 пунктов
\usepackage[russian]{babel} % задаёт русский как основной язык текста
\usepackage[T2A]{fontenc} % задаёт кириллическую кодировку шрифта
\usepackage{cmap} % обеспечивает нормальное копирование и поиск русского текста в pdf 
\usepackage[utf8]{inputenc} % определяет юникодную кодировку самого .tex-файла

\usepackage{geometry} % задаёт поля 
\geometry{left=3cm} % левое — 3 см
\geometry{right= 1.5cm} % правое — 1,5 см
\geometry{top=2cm} % верхнее — 2 см
\geometry{bottom=2cm} % нижнее — 2 см
\usepackage{setspace} \onehalfspacing % задаёт «полуторный» межстрочный интервал 
\usepackage{indentfirst} % автоматически добавляет отступ в каждый новый абзац

\begin{document} 
\begin{titlepage}
    \begin{center}
        {\bf  МИНОБРНАУКИ РОССИИ\\
        САНКТ-ПЕТЕРБУРГСКИЙ ГОСУДАРСТВЕННЫЙ\\
        ЭЛЕКТРОТЕХНИЧЕСКИЙ УНИВЕРСТИТЕТ\\
        <<ЛЭТИ>> ИМ. В. И. УЛЬЯНОВА (ЛЕНИНА)\\
    
        }
    \end{center}
    \vfill
        {
        \begin{center}
            КУРСОВАЯ РАБОТА\\
            по дисциплине <<Cхемотехника аналоговых устройств>>\\
            Тема: <<Проектирование усилителя на биполярных транзисторах>>\\
        \end{center}
        }
        \
    \vfill
        {\noindent\parbox{4cm}{Студент гр. 3114}  \hfill \parbox{3cm}{\rule{3cm}{0.15mm}} \hfill \parbox{4cm}{\raggedleft Злобин М. А.}\\}
        \parbox{4cm}{Преподаватель} \hfill \parbox{3cm}{\rule{3cm}{0.15mm}} \hfill \parbox{4cm}{\raggedleft Завьялов А. Е.} \\ 
        \center Санкт-Петербург
        
        2024
\end{titlepage}
\section{ЗАДАНИЕ НА КУРСОВУЮ РАБОТУ}
\noindent Студент Злобин М. А. \\
Группа 3114 \\
Тема работы: «Исследование прохождения сигналов через линейную активную электрическую цепь»
Исходные данные: вариант 8, схема 8\\
115 -- ИН $u_1$; 212 -- $R_2$; 325 -- $C_3$; 423 -- $R_4$; 525 -- $C_5$; ОУ -- 354, $k$;
724 -- $R_7$\\
Содержание пояснительной записки:\\
«Содержание», «Введение», 
«Нормирование параметров и переменных цепи», «Расчёт нулей и полюсов заданной функции передачи активной RC-цепи», «Поиск изображения входного одиночного импульса воздействия и вычисление реакции активной RC-цепи», «Вычисление переходной и импульсной характеристик активной RC-цепи», «Определение амплитудного и фазового спектров входного одиночного импульса», «Расчёт АЧХ и ФЧХ активной RC-цепи», «Амплитудный и фазовый спектры выходного одиночного импульса», «Определение амплитудного и фазового спектра периодического выходного сигнала», «Приближенный расчёт реакции цепи по спектру при периодическом воздействии», «Вычисление параметров активной электрической RC-цепи», 
«Выводы и заключение», «Список использованных источников»\\
Предполагаемый объем пояснительной записки:\\
Не менее \underline{\hspace{0.5cm}} страниц.\\
\\
Дата выдачи задания: \\
\\
Дата сдачи реферата: \\
\\
Дата защиты реферата: \\
\\
        {\noindent\parbox{4cm}{Студент гр. 3114}  \hfill \parbox{3cm}{\rule{3cm}{0.15mm}} \hfill \parbox{4cm}{\raggedleft Злобин М. А.}\\}
        \parbox{4cm}{Преподаватель} \hfill \parbox{3cm}{\rule{3cm}{0.15mm}} \hfill \parbox{4cm}{\raggedleft Завьялов А. Е.} \\ 
\newpage
\begin{center}
    {\bf АННОТАЦИЯ} 
\end{center}
    

    {
    Линейные электрические цепи играют ключевую роль в усилении и обработке сигналов, 
    проходящих через них. Для анализа таких цепей применяются методы преобразования Лапласа, 
    разложения в ряды Фурье и спектрального анализа. Изучение линейных цепей и сигналов, 
    которые через них проходят, позволяет предсказывать поведение схем при воздействии на них периодических сигналов.
    }
\begin{center}
    {\bf SUMMARY}
\end{center}


Linear electrical circuits are essential for amplifying and\\ processing the signals passing through them. 
Methods such as Laplace transform, Fourier series decomposition, and spectrum analysis are used to analyze these circuits. 
Studying linear circuits and the signals that pass through them allows for predicting the behavior of the circuit when subjected to certain periodic signals.
\newpage
\tableofcontents
\newpage
\section{ВВЕДЕНИЕ}
Цель курсовой работы – практическое освоение методов анализа искажений электрических сигналов, проходящих через линейные активные   RC~– цепи, а также рассмотрение вопросов проектирования активных RC – цепей по заданным передаточным функциям. 
В курсовой работе требуется выполнить следующие пункты: 
    \begin{enumerate}
        \item Найти по заданной передаточной функции реакцию активной RC-цепи при воздействии одиночного импульса; 
        \item Рассчитать переходную и импульсную характеристики активной цепи;  
        \item Найти спектральные характеристики аналогового входного сигнала и частотные характеристики цепи;  
        \item Вычислить 	установившуюся 	реакцию 	цепи 	при 	воздействии 
        периодической последовательности импульсов;  
        \item Рассчитать параметры элементов активной цепи по заданной передаточной функции.
    \end{enumerate}
    \newpage
\section{СПИСОК ИСПОЛЬЗОВАННЫХ ИСТОЧНИКОВ}
\begin{enumerate}
    \item Бычков Ю.А., Золотницкий В.М., Чернышев Э.П., Белянин А.Н. Основы теоретической электротехники: Учебное пособие. СПБ.: Изд-во “Лань”, 2008.  592 с.: ил. – (Учебники для вузов. Специальная литература). 
    \item Бычков Ю.А., Соловьева Е.Б., Чернышев Э.П. Курсовое проектирование по теоретической электротехниче: учеб. пособие в 2 ч. Ч. 1. СПб.: Изд-во СПбГЭТУ "ЛЭТИ", 2017. 109 с. 
    \item Иншаков Ю. М., Портной М. С. Исследование прохождения сигналов через линейную активную цепь: учеб.-метод. пособие. СПб.: Изд-во СПбГЭТУ "ЛЭТИ", 2024. 48 с. 
\end{enumerate}
\end{document}

