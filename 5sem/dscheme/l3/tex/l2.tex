\documentclass[a4paper,14pt ]{article} % можно использовать кегель 8-12, 14, 17 и 20 пунктов
\DeclareMathSizes{14}{14}{14}{14}
\usepackage{extsizes}
\usepackage{graphicx}
\graphicspath{{../}}
\usepackage[russian]{babel} % задаёт русский как основной язык текста
\usepackage[T2A]{fontenc} % задаёт кириллическую кодировку шрифта
\usepackage{cmap} % обеспечивает нормальное копирование и поиск русского текста в pdf 
\usepackage[utf8]{inputenc} % определяет юникодную кодировку самого .tex-файла
\setcounter{secnumdepth}{0}
\usepackage{mathptmx} 
\usepackage{fontspec}

\setmainfont{Times New Roman}


\usepackage{geometry} % задаёт поля 
\geometry{left=3cm} % левое — 3 см
\geometry{right= 1.5cm} % правое — 1,5 см
\geometry{top=2cm} % верхнее — 2 см
\geometry{bottom=2cm} % нижнее — 2 см
\usepackage{setspace} \onehalfspacing % задаёт «полуторный» межстрочный интервал 
\usepackage{indentfirst} % автоматически добавляет отступ в каждый новый абзац
\usepackage{amsmath,amsfonts,amssymb,amsthm,mathtools,mathtext, physics}
\usepackage{float}
\usepackage{array}
\usepackage{tabularx}
\usepackage{titlesec}
\usepackage{tikzPackets}
\usepackage{zref}
\usepackage{listings}
\titleformat{\section}{\centering\normalfont\bfseries}{\thesection.}{0.5em}{}
\titleformat{\subsection}{\centering\normalfont\bfseries}{\thesection.}{0.5em}{}
\titleformat{\subsubsection}{\centering\normalfont\bfseries}{\thesection.}{0.5em}{}
\setlength\parindent{1.25cm}
\setcounter{secnumdepth}{3}
\titlelabel{\thetitle. }
\begin{document}
\begin{titlepage}
    \thispagestyle{empty}
    \begin{center}
        {\bf  МИНОБРНАУКИ РОССИИ\\
        САНКТ-ПЕТЕРБУРГСКИЙ ГОСУДАРСТВЕННЫЙ\\
        ЭЛЕКТРОТЕХНИЧЕСКИЙ УНИВЕРСТИТЕТ\\
        <<ЛЭТИ>> ИМ. В. И. УЛЬЯНОВА (ЛЕНИНА)\\
        кафедра РС\\
    
        }
    \end{center}
    \vfill
        {
        \begin{center}
            \bfseries
            Отчет по лабораторной работе №3\\
            по дисциплине <<Cхемотехника цифровых устройств>>\\
            Тема: <<Исследование работы асинхронных RS- и DL-триггеров>>\\
            Вариант 8
        \end{center}
        }
        \
    \vfill
        {\noindent\parbox{4cm}{Студент гр. 3114}  \hfill \parbox{3cm}{\rule{3cm}{0.15mm}} \hfill \parbox{4cm}{\raggedleft Злобин М. А.}} \\\\
        \parbox{4cm}{Преподаватель} \hfill \parbox{3cm}{\rule{3cm}{0.15mm}} \hfill \parbox{5cm}{\raggedleft Овчинников М. А.} \\ 
        \center Санкт-Петербург
        
        2025
\end{titlepage}
\setcounter{page}{2}
    \section{Задание}
    \begin{enumerate}
         \item Собрать и схему RS-триггера в текстовом редакторе, изучить схему с помощью RTL-Viewer
         \item  Построить временные диаграммы с учетом задержек (время моделирования задать 450 нс) 
         \item Проделать все то же самое в графическом редакторе
         \item Cобрать схему асинхронного DL-триггера в текстовом редакторе, изучить схему с помощью RTL-Viewer
         \item Построить временную диаграмму с учетом задержек (время моделирования задать 300 нс)
    \end{enumerate}
    \section{асинхронный RS-триггер в текстовом редакторе}
    \lstinputlisting[numbers=left, label={list:rs}, caption=Описание асинхронного RS-триггера на языке Verilog, language=Verilog]{../s_r_ff.v}
    \begin{figure}[H]
      \center 
      \includegraphics[width=0.75\linewidth]{rtl1.png}      
      \label{fig:1}
      \caption{Схема асинхронного RS-триггера, созданного в текстовом редакторе}
    \end{figure}

    В схеме использован блок LATCH ("защелка"), блок по факту и реализует логику RS-триггера, переводя выходной сигнал q в 0 при
    сигнале на ACLR и в 1 при сигнале на PRESET. При это блоки "или" и "и" перед защелкой служат для обхода запрещенного состояния:
    сигналы на ACLR и PRESET не могут быть одновременно 1. На сигнал DATAIN подается 0, триггер от него не зависит.
    \begin{figure}[H]
      \center 
      \includegraphics[width=0.75\linewidth]{d1.png}      
      \label{fig:2}
      \caption{Временная диаграмма асинхронного RS-триггера, созданного в текстовом редакторе}
    \end{figure}
    На временной диаграмме видно, что в запрещенном режиме ($R=S=1$) сигнал $q=1$, сигнал $not\_q = 0$. 
    \section{асинхронный RS-триггер в графическом редакторе}
    Построим асинхронный RS-триггер в базисе "и-не" в графическом редакторе:
    \begin{figure}[H]
      \center 
      \includegraphics[width=0.75\linewidth]{sheme.png}      
      \label{fig:nand}
      \caption{Построение асинхронного RS-триггера в базисе и-не}
    \end{figure}
    Результат работы компонента RTL-Viewer:
    \begin{figure}[H]
      \center 
      \includegraphics[width=0.75\linewidth]{rtl2.png}      
      \label{fig:3}
      \caption{Схема асинхронного RS-триггера, созданного в графическом редакторе}
    \end{figure}
    \begin{figure}[H]
      \center 
      \includegraphics[width=0.75\linewidth]{d2.png}      
      \label{fig:5}
      \caption{Временная диаграмма асинхронного RS-триггера, созданного в графическом редакторе}
    \end{figure}
    Видно, что в запрещенном состянии сигналы $q$ и $not\_q$ не являются инверсными, оба 1.
\section{Aсинхронный DL-триггер}
    Описание асинхронного DL-триггера на языке Verilog:
    \lstinputlisting[numbers=left, caption=Описание асинхронного DL-триггера на языке Verilog, label={list:dl} language=Verilog]{../d_l_ff.v}
    \begin{figure}[H]
      \center 
      \includegraphics[width=0.75\linewidth]{rtldl.png}      
      \label{fig:s}
      \caption{Схема асинхронного DL-триггера}
    \end{figure}

    Также, как и в случае с RS-триггером, использльзован блок LATCH, выполняющий функции DL-триггера (входы 
    DATAIN и LATCH\_ENABLE соответствуют входам data и load). 
    На вход ACLR подается 0, т. к. выход триггера устанавливается в 0 комбинацией других сигналов.
    \begin{figure}[H]
      \center 
      \includegraphics[width=0.75\linewidth]{tdda2.png}      
      \label{fig:dld}
      \caption{Временная диаграмма асинхронного RS-триггера, созданного в графическом редакторе}
    \end{figure}
    При замене в 5 строчке @always(loar or data) на @always(data), изменений замечено не было:
    \begin{figure}[H]
      \center 
      \includegraphics[width=0.75\linewidth]{tdda2.png}      
      \label{fig:dld2}
      \caption{Временная диаграмма асинхронного RS-триггера}
    \end{figure}
    Схема RTL-Viewer также совпала.
    \section*{Вывод}
    В ходе работы были построены две модели RS-триггера: в текстовом и графическом редакторах, построены их временные диаграммы и схемы. Они отличаются лишь поведеним триггера в запрещенном состоянии.
    Также в текстовом редакторе был построен DL-триггер, построена его временная диаграмма и схемa. При измении списка чувствительности always его поведение не изменяется.
\end{document}
