\documentclass[a4paper,14pt ]{article} % можно использовать кегель 8-12, 14, 17 и 20 пунктов
\DeclareMathSizes{14}{14}{14}{14}
\usepackage{extsizes}
\usepackage{graphicx}
\graphicspath{{../}}
\usepackage[russian]{babel} % задаёт русский как основной язык текста
\usepackage[T2A]{fontenc} % задаёт кириллическую кодировку шрифта
\usepackage{cmap} % обеспечивает нормальное копирование и поиск русского текста в pdf 
\usepackage[utf8]{inputenc} % определяет юникодную кодировку самого .tex-файла
\setcounter{secnumdepth}{0}
\usepackage{mathptmx} 
\usepackage{fontspec}

\setmainfont{Times New Roman}


\usepackage{geometry} % задаёт поля 
\geometry{left=3cm} % левое — 3 см
\geometry{right= 1.5cm} % правое — 1,5 см
\geometry{top=2cm} % верхнее — 2 см
\geometry{bottom=2cm} % нижнее — 2 см
\usepackage{setspace} \onehalfspacing % задаёт «полуторный» межстрочный интервал 
\usepackage{indentfirst} % автоматически добавляет отступ в каждый новый абзац
\usepackage{amsmath,amsfonts,amssymb,amsthm,mathtools,mathtext, physics}
\usepackage{float}
\usepackage{array}
\usepackage{tabularx}
\usepackage{titlesec}
\usepackage{zref}
\titleformat{\section}{\centering\normalfont\bfseries}{\thesection.}{0.5em}{}
\titleformat{\subsection}{\centering\normalfont\bfseries}{\thesection.}{0.5em}{}
\titleformat{\subsubsection}{\centering\normalfont\bfseries}{\thesection.}{0.5em}{}
\setlength\parindent{1.25cm}
\setcounter{secnumdepth}{3}
\titlelabel{\thetitle. }
\begin{document}
\begin{titlepage}
    \thispagestyle{empty}
    \begin{center}
        {\bf  МИНОБРНАУКИ РОССИИ\\
        САНКТ-ПЕТЕРБУРГСКИЙ ГОСУДАРСТВЕННЫЙ\\
        ЭЛЕКТРОТЕХНИЧЕСКИЙ УНИВЕРСТИТЕТ\\
        <<ЛЭТИ>> ИМ. В. И. УЛЬЯНОВА (ЛЕНИНА)\\
        кафедра РС\\
    
        }
    \end{center}
    \vfill
        {
        \begin{center}
            \bfseries
            Отчет по лабораторной работе №1\\
            по дисциплине <<Cхемотехника цифровых устройств>>\\
            Тема: <<Построение схемы, заданной логическим выражением>>\\
            Вариант 8
        \end{center}
        }
        \
    \vfill
        {\noindent\parbox{4cm}{Студент гр. 3114}  \hfill \parbox{3cm}{\rule{3cm}{0.15mm}} \hfill \parbox{4cm}{\raggedleft Злобин М. А.}} \\\\
        \parbox{4cm}{Преподаватель} \hfill \parbox{3cm}{\rule{3cm}{0.15mm}} \hfill \parbox{5cm}{\raggedleft Овчинников М. А.} \\ 
        \center Санкт-Петербург
        
        2025
\end{titlepage}
\setcounter{page}{2}

   \section{Задание}
   Дано логическое выражение $\overline{x_3 \land x_2} \lor x_1$; 
   \begin{enumerate}
    \item Построить таблицу истинности для данного выражения
    \item Собрать в графическом редакторе схему цифрового устройства, 
    работа которого описывается данным логическим выражением.
    \item Показать результат работы компонента RTL Viewer.
    \item Построить временные диаграммы, иллюстрирующие работу устройства, при наличии и отсутствии задержек. 
   \end{enumerate}
   \section{Таблица истинности}
    Построим таблицу истинности для заданного логического выражения:
    Taблица 1. Таблица истинности для выражения $  \overline{x_3 \land x_2} \lor x_1 $.
   \begin{table}[H]
        \centering
        \begin{tabular}{|c|c|c|c|c|c|}
            \hline
            $x_1$ & $x_2$ & $x_3$ & $x_3 \land x_2 $ &$\overline{x_3 \land x_2}$ & $\overline{x_3 \land x_2} \lor x_1$\\
            \hline
             0 & 0 & 0 & 0 & 1 & 1 \\
             \hline
             0 & 0 & 1 & 0 & 1 & 1  \\
             \hline
             0 & 1 & 0 & 0 & 1 & 1 \\ 
             \hline
             0 & 1 & 1 & 1 & 0 & 0 \\
             \hline
             1 & 0 & 0 & 0 & 1 & 1\\
             \hline
             1 & 0 & 1 & 0 & 1 & 1 \\
             \hline
             1 & 1 & 0 & 0 & 1 & 1 \\
             \hline
             1 & 1 & 1 & 1 & 0 & 1 \\
             \hline
        \end{tabular}
        \label{tab:1}
   \end{table} 
\newpage
\section{Блок-схема}
Построим блок-схему с помощью редактора блок-схем в Quartus:
\begin{figure}[H]
    \centering
    \includegraphics[width=0.9\linewidth]{buildscheme.png}
    \caption{Схема логического выражения, построенная в редакторе блок-схем}
    \label{fig:0}
\end{figure}
\begin{figure}[H]
    \centering
    \includegraphics[width=0.5\linewidth]{scheme}

    \caption{Блок-схема исследуемого выражения, построенная RTL Viewer}
    \label{fig:1}
\end{figure}
Построим временные диаграммы с задержкой и без:
\begin{figure}[H]
    \centering
    \includegraphics[width=0.9\linewidth]{td}
    \caption{Временная диаграмма без задержки}
    \label{fig:2}
\end{figure}
    Выходной сигнал {\it out} соответствует последнему столбцу в таблице истинности.
\begin{figure}[H]
    \centering
    \includegraphics[width=0.9\linewidth]{tdd}
    \caption{Временная диаграмма с задержкой}
    \label{fig:3}
\end{figure}
    Подобная диагрмма моделирует реальную систему, в которой задержка выходного сигнала возникает из-за протекания 
    в ПЛИС переходных процессов, обусловленных паразитными емкостями {\it p-n} переходов.
\section*{Вывод}
В ходе выполнения работы в графическом редакторе программы \\ Quartus было составлено логическое выражение, которым 
был \space ''прошит'' \space ПЛИС семейства Cyclone IV E. 
Были проанализированы временные диаграммы, соответствующие идеальной модели и реальному 
устройству, в котором протекают переходные процессы.
\end{document}
