\documentclass[a4paper,14pt]{report} % можно использовать кегель 8-12, 14, 17 и 20 пунктов
\usepackage[russian]{babel} % задаёт русский как основной язык текста
\usepackage[T2A]{fontenc} % задаёт кириллическую кодировку шрифта
\usepackage{cmap} % обеспечивает нормальное копирование и поиск русского текста в pdf 
\usepackage[utf8]{inputenc} % определяет юникодную кодировку самого .tex-файла

\usepackage{geometry} % задаёт поля 
\geometry{left=3cm} % левое — 3 см
\geometry{right= 1.5cm} % правое — 1,5 см
\geometry{top=2cm} % верхнее — 2 см
\geometry{bottom=2cm} % нижнее — 2 см
\usepackage{setspace} \onehalfspacing % задаёт «полуторный» межстрочный интервал 
\usepackage{indentfirst} % автоматически добавляет отступ в каждый новый абзац

\begin{document} 
\begin{titlepage}
    \begin{center}
        {\bf  МИНОБРНАУКИ РОССИИ\\
        САНКТ-ПЕТЕРБУРГСКИЙ ГОСУДАРСТВЕННЫЙ\\
        ЭЛЕКТРОТЕХНИЧЕСКИЙ УНИВЕРСТИТЕТ\\
        <<ЛЭТИ>> ИМ. В. И. УЛЬЯНОВА (ЛЕНИНА)\\
        Кафедра ТОР }
    \end{center}
    \vfill
        {
        \begin{center}
            КУРСОВАЯ РАБОТА\\
            по дисциплине <<Теоретическая электродинамика>>\\
            Тема: <<Проектирование направленного ответвителя>>\\
        \end{center}
        }
        \
    \vfill
        {\noindent\parbox{4cm}{Студент гр. 3114}  \hfill \parbox{3cm}{\rule{3cm}{0.15mm}} \hfill \parbox{4cm}{\raggedleft Злобин М. А.}\\}
        \parbox{4cm}{Преподаватель} \hfill \parbox{3cm}{\rule{3cm}{0.15mm}} \hfill \parbox{4cm}{\raggedleft Балландович С. В.} \\ 
        \center Санкт-Петербург
        
        2024
\end{titlepage}
\section{ЗАДАНИЕ НА КУРСОВУЮ РАБОТУ}
\noindent Студент Злобин М. А. \\
Группа 3114 \\
Тема работы: «Проектирование направленного ответвителя»
Исходные данные: вариант 8\\
Тип НО: ШНО;\\
Частота: $2.3$ ГГц;\\
$S_{1,1} = -1.5$ дБ\\
Диэлектрик: Duroid 5870 \\
Содержание пояснительной записки:\\
<<Содержание>>, <<Введение>>, <<Расчет параметров ШНО>>, <<Схемотехническое моделирование ШНО>>,
<<Электродиначеское моделирование ШНО>>, <<Выводы>>
Предполагаемый объем пояснительной записки:\\
Не менее \underline{\hspace{0.5cm}} страниц.\\
\\
Дата выдачи задания: \\
\\
Дата сдачи реферата: \\
\\
Дата защиты реферата: \\
\\
        {\noindent\parbox{4cm}{Студент гр. 3114}  \hfill \parbox{3cm}{\rule{3cm}{0.15mm}} \hfill \parbox{4cm}{\raggedleft Злобин М. А.}\\}
        \parbox{4cm}{Преподаватель} \hfill \parbox{3cm}{\rule{3cm}{0.15mm}} \hfill \parbox{4cm}{\raggedleft Балландович С. В.} \\ 
\newpage
\begin{center}
    {\bf АННОТАЦИЯ} 
\end{center}
    

  { В схемах СВЧ часто возникает необходимость обеспечить разделение
 мощности для разветвления трактов или объединения 
 мощностей от нескольких источников. 
Направленный ответвитель — устройство для ответвления части 
электромагнитной энергии из основного канала передачи во вспомогательный. Направленный
ответвитель (НО) представляет собой два (иногда более) отрезка линий передачи,
связанных между собой определённым образом, основная линия называется первичной
, вспомогательная — вторичной. Для нормальной работы НО один из концов вторичной
линии (нерабочее плечо) должен быть заглушён согласованной нагрузкой,
со второго (рабочего плеча) снимается ответвлённый сигнал, в зависимости от
того, какую волну в первичной линии надо ответвить — падающую или отражённую.
Шлейфные НО состоят из двух отрезков полосковых линий передачи,
соединённых между собой с помощью двух и более шлейфов, длины и расстояния, между которыми равны четверти длины волны, определённой в полосковой линии передачи.     }
\begin{center}
    {\bf SUMMARY}
\end{center}

In microwave circuits, it is often necessary to provide
power separation for branching paths or combining 
 capacities from several sources. 
A directional coupler is a device for branching a portion
of electromagnetic energy from the main transmission channel to an auxiliary one. A directional
coupler (NO) is two (sometimes more) segments of transmission lines
connected in a certain way, the main line is called the primary
, the auxiliary is called the secondary. For normal operation, BUT one of the ends of the secondary
The line (non—working arm) must be damped by a coordinated load,
and a branch signal is removed from the second (working arm), depending on
which wave in the primary line needs to be branched - incident or reflected.
Loop cables consist of two segments of strip transmission lines
connected to each other by means of two or more loops, the lengths and distances between which are equal to a quarter of the wavelength defined in the strip transmission line.

\newpage
\tableofcontents
\newpage
\section{ВВЕДЕНИЕ}
Направленный ответвитель — устройство для ответвления части 
электромагнитной энергии из основного канала передачи во вспомогательный. Направленный
ответвитель (НО) представляет собой два (иногда более) отрезка линий передачи,
связанных между собой определённым образом, основная линия называется первичной
, вспомогательная — вторичной. Для нормальной работы НО один из концов вторичной
линии (нерабочее плечо) должен быть заглушён согласованной нагрузкой,
со второго (рабочего плеча) снимается ответвлённый сигнал, в зависимости от
того, какую волну в первичной линии надо ответвить — падающую или отражённую.
Шлейфные НО состоят из двух отрезков полосковых линий передачи,
соединённых между собой с помощью двух и более шлейфов, длины и расстояния, между которыми равны четверти длины волны, определённой в полосковой линии передачи. С увеличением числа шлейфов 
направленность и диапазонные характеристики шлейфового НО улучшаются.
    \newpage
    \section{Расчет параметров ШНО}
    Расчитаем значения сопротивлений полосков, соединённых шлейфами $Z_{\text{1}}$ и $Z_{\text{1}}$:
    Матрица рассеяния для двухшлейфного НО:

    \begin{equation}
      S = -\frac{1}{\sqrt{1+z_1^2}}
      \begin{vmatrix}
        0 & 0 & j & z_1 \\
        0 & 0 & z_1 & j \\
        j & z_1 & 9 & 0 \\
        z_1 & j & 0 & 0\\
      \end{vmatrix}
      ,
    \end{equation}
    
    где $z_i = \frac{Z_0}{Z_i}$. Условие согласования ($S(1,1) = 0$): 
    $z_1^2 =z_2^2 - 1$.

    Коэффициент деления мощности между плечами:
    \begin{equation}
    k = \frac{|S(3,1)|^2}{|S(4,1)|^2} = \frac1{z_1^2} = \frac{1}{z_2^2 - 1}
      \label{k}
    \end{equation}

    Найдем k:

    $S(1,2) = 10^{\frac{-1.5}{10}} = 0.701 \approx 0.7$. 

    Т. к.

    \begin{equation}
      |S(1,1)^2| + |S(1,2)^2| + |S(1,3)^2| = 1, 
    \end{equation}
    
    где $|S(1,1)| = 0$, получаем $S(1,3) = 0.3$.
    
    $k = \frac{7}{3}$, по формуле \ref{k} находим $Z_1$ и $Z_2$:
    \begin{equation}
      Z_1 = Z_0\sqrt{k}; \, Z_2 = Z_0\sqrt{\frac{k}{k+1}},
    \end{equation}

    $Z_1 = 76.376$ Ом;\\

    $Z_2 = 41.83$ Ом.
\section{Электротехническое моделирование}
  Электротехническое моделирование будет выполнено в пакете AWR Design Environment.\\
  
  С помощью компонента TXLine расчитаем физические параметры $\frac{\lambda}4$-отрезков:
  \begin{figure}[H]
    \centering
    \includegraphics[width=0.8\linewidth]{P50.png}
    \caption{Расчет параметры 50-омного полоска}
  \end{figure}
  \begin{figure}[H]
    \centering
    \includegraphics[width=0.8\linewidth]{P0.png}
    \caption{Расчет параметры 76.376-омного полоска}
  \end{figure}
  \begin{figure}[H]
    \centering
    \includegraphics[width=0.8\linewidth]{P1.png}
    \caption{Расчет параметров 41.83-омного полоска}
  \end{figure}
  Постоим схему ШНО:
  \begin{figure}[H]
    \centering
    \includegraphics[width=1\linewidth]{c-sheme.png}
    \caption{Схема ШНО}
  \end{figure}
  \begin{figure}[H]
    \centering
    \includegraphics[width=0.8\linewidth]{layout.png}
    \caption{Модель ШНО}
  \end{figure}
  Построим графики $S(1,1)$, $S(1,2)$, $S(1,3)$, $S(1,4)$ в диапазоне 1.8-2.8 ГГц с шагом 0.01 ГГц:
  \begin{figure}[H]
    \centering
    \includegraphics[width=0.8\linewidth]{s0.png} 
    \caption{Модули коэффициентов отражения, дБ}
  \end{figure}

  \begin{figure}[H]
    \centering
    \includegraphics[width=0.8\linewidth]{phase.png} e
    \caption{Фазы коэффициентов отражения}
  \end{figure}
  На графиках видно, что частота согласования и величина  $S(2,1)$ отличаются
  от заданных. Подгоним параметры длинной линии: изменяя величину длины
  полоска можно смещать положение резонанса, а меняя ширины W2, W3 можно
  менять модули коэффициентов отражения:
  \begin{figure}[H]
    \centering
    \includegraphics[width=0.8\linewidth]{Pch.png} e
    \caption{Итоговые параметры}
  \end{figure}
  \begin{figure}[H]
    \centering
    \includegraphics[width=0.8\linewidth]{s1.png} e
    \caption{Моудли коэффициентов отражения после подгонки}
  \end{figure}
  \begin{figure}[H]
    \centering
    \includegraphics[width=0.8\linewidth]{phase.png} e
    \caption{Фазы коэффициентов после подгонки}
  \end{figure}
  \section{Электродинамическое моделирование}
  Создадим элемент EM Structure и зададим параметры параметры:
  \begin{figure}[H]
    \centering
    \includegraphics[width=0.8\linewidth]{EMset.png}
    \caption{Размеры EM Structure, соответствующие размерам ШНО}
  \end{figure}
  \begin{figure}[H]
    \centering
    \includegraphics[width=0.8\linewidth]{EMdiel.png}
    \caption{Параметры диэлектрика в EM Structure}
  \end{figure}
  Впишем построенный ранее ШНО в структуру и зададим выходные порты:
  \begin{figure}[H]
    \centering
    \includegraphics[width=0.8\linewidth]{EMscheme.png}
    \caption{ШНО в EM Structure}
  \end{figure}
  Электродинамическое моделирование проведем для шага сетки,
  равного 0.1 мм, и диапазона частот 1.8-2.8 ГГц с шагом 0.1 ГГц.
  \begin{figure}[H]
    \centering
    \includegraphics[width=0.8\linewidth]{EMphase.png}
    \caption{Модули коэффициентов отражения, дБ}
  \end{figure}
  \begin{figure}[H]
    \centering
    \includegraphics[width=0.8\linewidth]{EMmagn.png}
    \caption{Фазы коэффициентов отражения}
  \end{figure}
  Уровень согласования значительно хуже, чем при схемотехническом моделирование,
  но все равно ниже уровня -30 дБ
  \section{Выводы}
В ходе работы были выполнены:
\begin{enumerate}
  \item Расчет параметров ШНО;
  \item Схемотехническое моделирование ШНО;
  \item Электродинамическое моделирование ШНО;
\end{enumerate}

Шлейфный направленный ответвитель с неравным делением мощности (7/3), разработанный в данной работе, соответствует заданным требованиям на этапе схемотехнического моделирования, но демонстрирует некоторые расхождения в электродинамическом моделировании. \newpage
\section{СПИСОК ИСПОЛЬЗОВАННЫХ ИСТОЧНИКОВ}
\begin{enumerate}
  \item Устройства СВЧ: конспект лекций. СПб.: Изд-во СПбГЭТУ <<ЛЭТИ>>, 2014. 92 с.
  \item Компьютерное проектирование устройств СВЧ. СПб. Изд-во СПбГЭТУ, <<ЛЭТИ>>, 2010. 120 с.
\end{enumerate}

\end{document}
\end{document}

