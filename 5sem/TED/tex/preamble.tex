\documentclass[a4paper,14pt ]{article} % можно использовать кегель 8-12, 14, 17 и 20 пунктов
\DeclareMathSizes{14}{14}{14}{14}
\usepackage{extsizes}
\usepackage{graphicx}
\graphicspath{{../}}
\usepackage[russian]{babel} % задаёт русский как основной язык текста
\usepackage[T2A]{fontenc} % задаёт кириллическую кодировку шрифта
\usepackage{cmap} % обеспечивает нормальное копирование и поиск русского текста в pdf 
\usepackage[utf8]{inputenc} % определяет юникодную кодировку самого .tex-файла
\setcounter{secnumdepth}{0}
\usepackage{mathptmx} 
\usepackage{fontspec}
\usepackage{subcaption}

\usepackage{caption}
\usepackage[labelsep=endash, name=Рисунок]{subcaption}
\usepackage[labelsep=endash]{caption}

\captionsetup[figure]{name=Рисунок}
\captionsetup[subfigure]{labelformat=simple}
\setmainfont{Times New Roman}


\usepackage{geometry} % задаёт поля 
\geometry{left=3cm} % левое — 3 см
\geometry{right= 1.5cm} % правое — 1,5 см
\geometry{top=2cm} % верхнее — 2 см
\geometry{bottom=2cm} % нижнее — 2 см
\usepackage{setspace} \onehalfspacing % задаёт «полуторный» межстрочный интервал 
\usepackage{indentfirst} % автоматически добавляет отступ в каждый новый абзац
\usepackage{amsmath,amsfonts,amssymb,amsthm,mathtools,mathtext, physics}
\usepackage{float}
\usepackage{array}
\usepackage{tabularx}
\usepackage{titlesec}
\usepackage{zref}
\titleformat{\section}{\centering\normalfont\bfseries}{\thesection.}{0.5em}{}
\titleformat{\subsection}{\centering\normalfont\bfseries}{\thesubsection.}{0.5em}{}  % Исправлено
\titleformat{\subsubsection}{\centering\normalfont\bfseries}{\thesubsubsection.}{0.5em}{}
\setlength\parindent{1.25cm}
\setcounter{secnumdepth}{3}

