\documentclass[a4paper,14pt ]{article} % можно использовать кегель 8-12, 14, 17 и 20 пунктов
\DeclareMathSizes{14}{14}{14}{14}
\usepackage{extsizes}
\usepackage{graphicx}
\graphicspath{{../}}
\usepackage[russian]{babel} % задаёт русский как основной язык текста
\usepackage[T2A]{fontenc} % задаёт кириллическую кодировку шрифта
\usepackage{cmap} % обеспечивает нормальное копирование и поиск русского текста в pdf 
\usepackage[utf8]{inputenc} % определяет юникодную кодировку самого .tex-файла
\setcounter{secnumdepth}{0}
\usepackage{mathptmx} 
\usepackage{fontspec}

\setmainfont{Times New Roman}


\usepackage{geometry} % задаёт поля 
\geometry{left=3cm} % левое — 3 см
\geometry{right= 1.5cm} % правое — 1,5 см
\geometry{top=2cm} % верхнее — 2 см
\geometry{bottom=2cm} % нижнее — 2 см
\usepackage{setspace} \onehalfspacing % задаёт «полуторный» межстрочный интервал 
\usepackage{indentfirst} % автоматически добавляет отступ в каждый новый абзац
\usepackage{amsmath,amsfonts,amssymb,amsthm,mathtools,mathtext, physics}
\usepackage{float}
\usepackage{array}
\usepackage{tabularx}
\usepackage{titlesec}
\usepackage{tikzPackets}
\usepackage{zref}
\usepackage{listings}
\titleformat{\section}{\centering\normalfont\bfseries}{\thesection.}{0.5em}{}
\titleformat{\subsection}{\centering\normalfont\bfseries}{\thesection.}{0.5em}{}
\titleformat{\subsubsection}{\centering\normalfont\bfseries}{\thesection.}{0.5em}{}
\setlength\parindent{1.25cm}
\setcounter{secnumdepth}{3}
\titlelabel{\thetitle. }
\begin{document}
\begin{titlepage}
    \thispagestyle{empty}
    \begin{center}
        {\bf  МИНОБРНАУКИ РОССИИ\\
        САНКТ-ПЕТЕРБУРГСКИЙ ГОСУДАРСТВЕННЫЙ\\
        ЭЛЕКТРОТЕХНИЧЕСКИЙ УНИВЕРСТИТЕТ\\
        <<ЛЭТИ>> ИМ. В. И. УЛЬЯНОВА (ЛЕНИНА)\\
        кафедра РС\\
    
        }
    \end{center}
    \vfill
        {
        \begin{center}
            \bfseries
            Отчет по лабораторной работе №2\\
            по дисциплине <<Cхемотехника цифровых устройств>>\\
            Тема: <<Построение схемы заданной диаграммой Вейча>>\\
            Вариант 8
        \end{center}
        }
        \
    \vfill
        {\noindent\parbox{4cm}{Студент гр. 3114}  \hfill \parbox{3cm}{\rule{3cm}{0.15mm}} \hfill \parbox{4cm}{\raggedleft Зинковский Ю. А.}} \\\\
        \parbox{4cm}{Преподаватель} \hfill \parbox{3cm}{\rule{3cm}{0.15mm}} \hfill \parbox{5cm}{\raggedleft Овчинников М. А.} \\ 
        \center Санкт-Петербург
        
        2025
\end{titlepage}
\setcounter{page}{2}
    \begin{figure}[H] 
        \centering
        \begin{tikzpicture}<br>
        %\draw[help lines] (-1, -1) grid (6, 6);
        \draw (0,0) grid (4,4); 
        \draw (0.5, 3.5) node {1}; 
        \draw (1.5, 3.5) node {1}; 
        \draw (1.5, 3.5 - 1) node {1}; 
        \draw (1.5 - 1, 3.5 - 1) node {1}; 
        \draw (1.5 + 2, 3.5) node {1}; 
        \draw (1.5 + 1, 3.5) node {1}; 
        \draw (1.5 + 1, 3.5) node {1}; 
        \draw (3.5, 0.5) node {1}; 
        \draw (2.5, 0.5) node {1}; 
        \draw (0.5, 0.5) node {0}; 

        \draw (2.5, 2.5) node {0};
        \draw (3.5, 2.5) node {0};
        \draw (.5, 1.5) node {0};
        \draw (1.5, 1.5) node {0};
        \draw (2.5, 1.5) node {1};
        \draw (3.5, 1.5) node {1};
        \draw (1.5, 0.5) node {0};
        
        
        
        \draw (0,4.3) --node[midway, above]{$x_4$} (2,4.3);
        \draw (-0.3,4) --node[midway, left]{$x_2$} (-0.3,2);
        \draw (1,-0.3) --node[midway, below]{$x_3$} (3,-0.3);
        \draw (4.3,1) --node[midway, right]{$x_1$} (4.3,3);
    \end{tikzpicture}
    \caption{Диаграмма Вейча}
    \end{figure} 

    \section{Построение МКНФ}
    \begin{figure}[H] 
        \centering
        \begin{tikzpicture}<br>
        %\draw[help lines] (-1, -1) grid (6, 6);
        \draw (0,0) grid (4,4); 
        \draw (0.5, 3.5) node {1}; 
        \draw (1.5, 3.5) node {1}; 
        \draw (1.5, 3.5 - 1) node {1}; 
        \draw (1.5 - 1, 3.5 - 1) node {1}; 
        \draw (1.5 + 2, 3.5) node {1}; 
        \draw (1.5 + 1, 3.5) node {1}; 
        \draw (1.5 + 1, 3.5) node {1}; 
        \draw (3.5, 0.5) node {1}; 
        \draw (2.5, 0.5) node {1}; 
        \draw (0.5, 0.5) node {0}; 

        \draw (2.5, 2.5) node {0};
        \draw (3.5, 2.5) node {0};
        \draw (.5, 1.5) node {0};
        \draw (1.5, 1.5) node {0};
        \draw (2.5, 1.5) node {1};
        \draw (3.5, 1.5) node {1};
        \draw (1.5, 0.5) node {0};
        
        
        \draw (0,4.3) --node[midway, above]{$x_4$} (2,4.3);
        \draw (-0.3,4) --node[midway, left]{$x_2$} (-0.3,2);
        \draw (1,-0.3) --node[midway, below]{$x_3$} (3,-0.3);
        \draw (4.3,1) --node[midway, right]{$x_1$} (4.3,3);
        


       \draw (2.1,2.9) -- (2.1, 2.1) -- (3.9, 2.1) -- (3.9, 2.9) -- (2.1, 2.9);
        \draw (0.1, 0.1) -- (0.1, 1.9) -- (1.9, 1.9) -- (1.9, 0.1) -- (0.1, 0.1);
        %\draw (0.1, 0.9) rectangle (0.9, 0); 
        
        \draw (0,4.3) --node[midway, above]{$x_4$} (2,4.3);
        \draw (-0.3,4) --node[midway, left]{$x_2$} (-0.3,2);
        \draw (1,-0.3) --node[midway, below]{$x_3$} (3,-0.3);
        \draw (4.3,1) --node[midway, right]{$x_1$} (4.3,3);
    \end{tikzpicture}
    \caption{Объединение кубов}
        
    \end{figure} 
    Запишем MКНФ:
    \begin{equation}
      y = (\overline{x_2} \lor x_4) \land (\overline{x_4} \lor x_1 \lor x_2)  
        \label{eq:1}
    \end{equation} 
    \section{Таблица истинности}
    Построим таблицу истинности для найденной МКНФ:
    \begin{table}[H]
      \caption{Таблица истинности для логической функции Таблица истинности для логической функции $y = (\overline{x_2} \lor x_4) \land (\overline{x_4} \lor x_1 \lor x_2)$}

\begin{tabular}{|c|c|c|c|c|c|c|}
\hline
$x_1$ & $x_2$ & $x_3$ & $x_4$ & $\overline{x_2} \lor x_4$ & $\overline{x_4} \lor x_1 \lor x_2$ & $(\overline{x_2} \lor x_4) \land (\overline{x_4} \lor x_1 \lor x_2)$ \\
\hline
0 & 0 & 0 & 0 & 1 & 1 & 1 \\
\hline
0 & 0 & 0 & 1 & 1 & 0 & 0 \\
\hline
0 & 0 & 1 & 0 & 1 & 1 & 1 \\
\hline
0 & 0 & 1 & 1 & 1 & 0 & 0 \\
\hline
0 & 1 & 0 & 0 & 0 & 1 & 0 \\
\hline
0 & 1 & 0 & 1 & 1 & 1 & 1 \\
\hline
0 & 1 & 1 & 0 & 0 & 1 & 0 \\
\hline
0 & 1 & 1 & 1 & 1 & 1 & 1 \\
\hline
1 & 0 & 0 & 0 & 1 & 1 & 1 \\
\hline
1 & 0 & 0 & 1 & 1 & 1 & 1 \\
\hline
1 & 0 & 1 & 0 & 1 & 1 & 1 \\
\hline
1 & 0 & 1 & 1 & 1 & 1 & 1 \\
\hline
1 & 1 & 0 & 0 & 0 & 1 & 0 \\
\hline
1 & 1 & 0 & 1 & 1 & 1 & 1 \\
\hline
1 & 1 & 1 & 0 & 0 & 1 & 0 \\
\hline
1 & 1 & 1 & 1 & 1 & 1 & 1 \\
\hline
\end{tabular}
        \label{tab:truth_table}
    
    \end{table}
    \newpage
    \section{Блок-диаграмма}
    Построим блок-схему устройства, реализующую МКНФ используя редактор блок-схем Quartus:
    \begin{figure}[H]
        \centering
        \includegraphics[width=0.9\linewidth]{blockscheme.png}
        \caption{Блок диаграмма, построенная в редакторе Quartus}
        \label{fig:1}
    \end{figure}
    Воспользуемся компонентом RTL Viewer:
    \begin{figure}[H]
        \centering
        \includegraphics[width=0.7\linewidth]{td.png}
        \caption{Блок диаграмма построенная RTL viewer}
        \label{fig:1}
    \end{figure}
    \section{Моделирование с помощью Verilog}
    Cмоделируем устройство, описываемой МКНФ (1) с помощью языка описания аппарутру Verilog.
    Код Verilog:
    \lstinputlisting[language=Verilog]{../quartus/simple.v}
    \begin{figure}[H]
        \centering
        \includegraphics[width=0.7\linewidth]{td2.png}
        \caption{Блок диаграмма построенная RTL viewer}
        \label{fig:2}
    \end{figure}
    Схемы \ref{fig:1} и \ref{fig:2} отличаются расположением переменных и вентилей, но 
    описывают одно и то же выражение.
    \section{Временные диаграммы}
    Построим временные диаграммы c задержкой и без для устройства, описываемого ККНФ (1):
    \begin{figure}[H]
        \centering
        \includegraphics[width=0.8\linewidth]{time}
        \caption{Временная диаграмма без задержки на выходе}
        \label{fig:3}
    \end{figure}
    Расположение переменных и значения выхода {\it y1} соответствуют таблице истинности.
    \begin{figure}[H]
        \centering
        \includegraphics[width=0.8\linewidth]{timed}
        \caption{Временная диаграмма c задержкой на выходе}
        \label{fig:4}
    \end{figure}
    Временная диаграмма с задержкой из-за протекания в ПЛИС переходных процессов.
    \section*{Вывод}
    В ходе работы по диаграмме Вейча была составлена МКНФ и построена соответствующая таблица истинности.
    Двумя способами было смоделировано цифровое устройство, описываемое работой данной МКНФ: с помощью графического редактора
    и с помощью языка описания аппаратуры Verlog. Результаты работы компонента RTL Viewer в двух случаях представлены на рисунках
    \ref{fig:1} и \ref{fig:2}. Для модели на Verilog были получены временные диаграммы (\ref{fig:3}, \ref{fig:4}), \space отражающие как идеальный случай, так и 
    реальное устройство с наличием задержки на выходе.

\end{document}
