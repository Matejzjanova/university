
\documentclass[a4paper,14pt ]{article} % можно использовать кегель 8-12, 14, 17 и 20 пунктов
\DeclareMathSizes{14}{14}{14}{14}
\usepackage{extsizes}
\usepackage{graphicx}
\graphicspath{{/home/misha/VUZ/TOE/laby/l9}}
\usepackage[russian]{babel} % задаёт русский как основной язык текста
\usepackage[T2A]{fontenc} % задаёт кириллическую кодировку шрифта
\usepackage{cmap} % обеспечивает нормальное копирование и поиск русского текста в pdf 
\usepackage[utf8]{inputenc} % определяет юникодную кодировку самого .tex-файла
\setcounter{secnumdepth}{0}
\usepackage{geometry} % задаёт поля 
\geometry{left=3cm} % левое — 3 см
\geometry{right= 1.5cm} % правое — 1,5 см
\geometry{top=2cm} % верхнее — 2 см
\geometry{bottom=2cm} % нижнее — 2 см
\usepackage{setspace} \onehalfspacing % задаёт «полуторный» межстрочный интервал 
\usepackage{indentfirst} % автоматически добавляет отступ в каждый новый абзац
\usepackage{amsmath,amsfonts,amssymb,amsthm,mathtools,mathtext, physics}
\usepackage{float}
\usepackage{array}
\usepackage{tabularx}
\usepackage{titlesec}
\titleformat{\section}{\centering\normalfont\bfseries}{\thesection}{1em}{}
\titleformat{\subsection}{\centering\normalfont\bfseries}{\thesection}{1em}{}
\titleformat{\subsubsection}{\centering\normalfont\bfseries}{\thesection}{1em}{}
\setlength\parindent{1.25cm}
\begin{document} 
\begin{titlepage}
    \begin{center}
        {\bf  МИНОБРНАУКИ РОССИИ\\
        САНКТ-ПЕТЕРБУРГСКИЙ ГОСУДАРСТВЕННЫЙ\\
        ЭЛЕКТРОТЕХНИЧЕСКИЙ УНИВЕРСТИТЕТ\\
        <<ЛЭТИ>> ИМ. В. И. УЛЬЯНОВА (ЛЕНИНА)\\
    
        }
    \end{center}
    \vfill
        {
        \begin{center}
            \bfseries
            ЛАБОРАТОРНАЯ РАБОТА №6\\
            по дисциплине <<Теоретические основы электротехники>>\\
            Тема: <<ИССЛЕДОВАНИЕ ИНДУКТИВНО СВЯЗАННЫХ ЦЕПЕЙ>>\\
        \end{center}
        }
        \
    \vfill
        {\noindent\parbox{4cm}{Студенты гр. 3114}  \hfill \parbox{3cm}{\rule{3cm}{0.15mm}} \hfill \parbox{4cm}{\raggedleft Злобин М. А.\\ Федулова Л. В. \\ Раузер А. А.}} \\\\
        \parbox{4cm}{Преподаватель} \hfill \parbox{3cm}{\rule{3cm}{0.15mm}} \hfill \parbox{4cm}{\raggedleft Лановенко Е. В.} \\ 
        \center Санкт-Петербург
        
        2025
\end{titlepage}
Цель работы: практическое ознакомление с синусоидальными режимами в простых RL-, RC- и RLC-цепях.
При анализе электрических цепей в установившемся синусоидальном режиме важно твёрдо усвоить амплитудные и фазовые соотношения между токами и напряжениями элементов цепи. Необходимо помнить, что ток в 
R-элементе совпадает по фазе с напряжением, ток в L-элементе отстаёт, а в 
C-элементе опережает напряжение на четверть периода (90°).
	Следует учитывать, что комплексные сопротивления индуктивности и ёмкости есть функции частоты:

    \begin{equation}
        \begin{cases}
            Z_L = j\omega L\\
            Z_c = \frac{1}{j\omega c} = \frac{1}{\omega C}e^{-j90}
        \end{cases}
    \end{equation}
    Функциями частоты являются, следовательно, и комплексные сопротивления RL-, RC- и RLC-цепей. Так, для RLC-цепи, изображенной на рис. 1, в, комплексное сопротивление 
    \begin{equation}
        Z = \frac{\dot{U}}{\dot{I}} = R + Z_L + Z_C = R + j\left[\omega L - \frac{j}{\omega C}\right]
    \end{equation}
    \begin{figure}[H]
        \centering
        \includegraphics[width=1\linewidth]{cheme}
        \caption{RL-, RC-, RLC-цепь}
        \label{fig:1}
    \end{figure}
\section{Исследование установившегося синусоидального режима в RC-цепи}
\end{document}